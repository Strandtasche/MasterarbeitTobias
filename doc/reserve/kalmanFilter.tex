\section{Das Kalman-Filter}
\todo[inline]{Entscheiden, ob diese Section komplett weg soll}
Als Kalman-Filter bezeichnet man ein mathematisches Verfahren mit dem Messfehler in realen Messwerten reduziert werden können und nicht messbare Systemgrößen geschätzt werden können. 


\todo{vergangene, aktuelle und zukünftige Systemzustände schätzen}
\todo{Einschränkung Linearität (Extended Kalman) und Gauß rauschen}

Der Zustand des Systems zum Zeitschritt \(t\) wird als \(y_t\) und die Messung im Zeitschritt \(t\) als \(z_t\) bezeichnet.

\begin{equation}
	y_t = A y_{t-1} + w, 	w \sim N(0, Q)
\end{equation}

\begin{equation}
	z_t = H y_{t} + v, 	v \sim N(0, R)
\end{equation}

Dabei ist \(A\) die Zustandsübergangsmatrix, die den Übergang von einem Zustand in den nächsten beschreibt.
\(H\) ist die Messmatrix, die beschreibt wie Messungen aus dem Zustand entstehen und Q und R sind die Kovarianzmatrizen des Systemrauschens beziehungsweise des Messrauschens. 

Das Kalman-Filter funktioniert mittels abwechselnd ausgeführter \textit{predict} und \textit{update} Schritte.

\begin{equation}
\hat{y}'_t = A \hat{y}'_{t-1}
\end{equation}

\begin{equation}
	\hat{P}'_t = A \hat{P}'_{t-1} A^\textit{T} + Q
\end{equation}




