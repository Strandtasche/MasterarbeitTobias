\section{Stand der Technik}

Notizen:
\todo[inline]{Das aller aller dickste TODO}
\color{blue}
Wie wird das, was ich dann mit Neuronalen Netzen machen möchte aktuell gemacht?


Teilaspekt vom Tracksort Multi-Targettracking: Bewegungsprädiktion.
Primär Florian's Dissertation: Kapitel 4.


\begin{itemize}
    \item Im Rahmen dieser Arbeit es geht um die Prädiktion der Bewegung von Schüttgutpartikeln.
    \item "Bewegungsmodell" bei Zeilenkamera: Identical Delay, weil nur eine Beobachtung.
    \item Modelle erst seit Flächenkamera notwendig.
    \item Vorraussetzung: Trackassoziationen.
    \item Aber: wir nehmen dieses Problem als gelöst an.
    \item In der Realität ist es das nicht (MA Tobi K), reale daten sind nicht zwingend 100\% genau, aber wir arbeiten mal damit.
    \item in Florians Diss: mehrere Modelle beschrieben zum prädizieren:
    \item CV und CA, und was die dahinterstehenden Annahmen sind.
    \item CV: Konstante geschwindigkeit - Perfekte Beruhigung
    \item CA: Konstante Beschleuningung - Geschwindigkeit ändert sich Konstante
    \item Dazu kommen noch Szenario spezifische Modelle, bei denen gezeigt wurden, dass sie für das Szenario besser sind.
    \item CVBC, IA - upgrades.
    \item Vorgriff ins evaluationskapitel? - dort wird dann behandelt wie man sie Mathematisch beschreibt. 
\end{itemize}

\color{black}


Im Rahmen dieser Arbeit geht es um die Prädiktion der Bewegungsprädiktion von Schüttgutpartikeln.
Der Einsatz von verschiedenen Bewegungsmodellen ist erst für Schüttgutsortierer mit Flächenkamera sinnvoll.
Eine Zeilenkamera liefert nur einen einzelnen Datenpunkt bezüglich Zeit und Position eines Partikels.
In~\cite{Pfaff2018} wird unter anderem ein Bewegungsmodell beschrieben, das das Verhalten eines Schüttgutsortierers mit Zeilenkamera emuliert.
Wie in schon in Abbildung~\ref{fig:predMissed} dargestellt wurde, muss angenommen werden, dass es zu keinerlei Bewegung orthogonal zur Transportrichtung kommt.
Für die Prädiktion des Zeitpunkts wird die durchschnittliche Zeit bestimmt, die ein Partikel von der Position der Zeilenkamera zur Position des Druckluftdüsenarrays benötigt 
und diese als konstanter Offset für jede Partikeldetektion angenommen.

Um den Separationsprozess durch den Einsatz von prädiktiven Tracking Methoden und Bewegungsmodellen zu verbessern ist eine Assoziation 
der beobachteten Partikelpositionen zu tatsächlichen Tracks notwendig. 
Im Rahmen dieser Arbeit wird dieses Problem nicht betrachtet. 
Es wird direkt mit den assozierten Trackdaten gearbeitet, 
obwohl dieser Assoziationsprozess noch Gegenstand aktueller Forschung ist.
\todo{hier schon erwähnen, dass die Assoziation auf den selbst gesammelten Daten vielleicht flawed ist?}

Die grundlegenden Bewegungsmodelle, die in \cite{Pfaff2018} beschrieben werden sind 
einerseits das Constant Velocity Modell und andererseits das Constant Acceleration Modell.


