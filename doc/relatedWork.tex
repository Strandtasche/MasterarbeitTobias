\section{Stand der Technik}
\label{cap:relWork}

Notizen:
\todo[inline]{Das aller aller dickste TODO}
\color{blue}
Wie wird das, was ich dann mit Neuronalen Netzen machen möchte aktuell gemacht?


Teilaspekt vom Tracksort Multi-Targettracking: Bewegungsprädiktion.
Primär Florian's Dissertation: Kapitel 4.


\begin{itemize}
    \item Im Rahmen dieser Arbeit es geht um die Prädiktion der Bewegung von Schüttgutpartikeln.
    \item "Bewegungsmodell" bei Zeilenkamera: Identical Delay, weil nur eine Beobachtung.
    \item Modelle erst seit Flächenkamera notwendig.
    \item Vorraussetzung: Trackassoziationen.
    \item Aber: wir nehmen dieses Problem als gelöst an.
    \item In der Realität ist es das nicht (MA Tobi K), reale daten sind nicht zwingend 100\% genau, aber wir arbeiten mal damit.
    \item in Florians Diss: mehrere Modelle beschrieben zum prädizieren:
    \item CV und CA, und was die dahinterstehenden Annahmen sind.
    \item CV: Konstante Geschwindigkeit - Perfekte Beruhigung
    \item CA: Konstante Beschleuningung - Geschwindigkeit ändert sich Konstante
    \item Dazu kommen noch Szenario spezifische Modelle, bei denen gezeigt wurden, dass sie für das Szenario besser sind.
    \item CVBC, IA - upgrades.
    \item Vorgriff ins evaluationskapitel? - dort wird dann behandelt wie man sie Mathematisch beschreibt. 
\end{itemize}

\color{black}


Im Rahmen dieser Arbeit geht es um die Prädiktion der Bewegungsprädiktion von Schüttgutpartikeln.
Der Einsatz von verschiedenen Bewegungsmodellen ist erst für Schüttgutsortierer mit Flächenkamera sinnvoll.
Eine Zeilenkamera liefert nur einen einzelnen Datenpunkt bezüglich Zeit und Position eines Partikels.
In~\cite{Pfaff2018} wird unter anderem ein Bewegungsmodell beschrieben, das das Verhalten eines Schüttgutsortierers mit Zeilenkamera emuliert.
Wie in schon in Abbildung~\ref{fig:predMissed} dargestellt wurde, muss angenommen werden, dass es zu keinerlei Bewegung orthogonal zur Transportrichtung kommt.
Für die Prädiktion des Zeitpunkts wird die durchschnittliche Zeit bestimmt, die ein Partikel von der Position der Zeilenkamera zur Position des Druckluftdüsenarrays benötigt 
und diese als konstanter Offset für jede Partikeldetektion angenommen.

Um den Separationsprozess durch den Einsatz von prädiktiven Tracking Methoden und Bewegungsmodellen zu verbessern ist eine Assoziation 
der beobachteten Partikelpositionen zu tatsächlichen Tracks notwendig. 
Im Rahmen dieser Arbeit wird dieses Problem nicht betrachtet. 
Es wird direkt mit den assozierten Trackdaten gearbeitet, 
obwohl dieser Assoziationsprozess noch Gegenstand aktueller Forschung ist.
\todo{hier schon erwähnen, dass die Assoziation auf den selbst gesammelten Daten vielleicht flawed ist?}

Die grundlegenden Bewegungsmodelle, die in \cite{Pfaff2018} beschrieben werden sind 
einerseits das Constant Velocity Modell und andererseits das Constant Acceleration Modell.
Das Constant Velocity Modell prädiziert die Bewegung eines Teilchens unter der Annahme, dass es sich mit einer konstanten Geschwindigkeit bewegt.
Basierend auf den letzten zwei bekannten Positionen des Partikels wird dessen Geschwindigkeit entlang der beiden Achsen bestimmt 
und davon die zukünftige Bewegung abgeleitet.
Wenn bei einem Bandsortierer jedoch zum Beispiel das Förderband nicht lang genug ist um das Schüttgut zu beruhigen kann es sein, 
dass die Teilchen eine Beschleuningung haben, die nicht 0 ist.
Das Constant Velocity Modell dahingegen prädiziert die Bewegung des Teilchens unter der Annahme, dass es sich mit einer konstanten Beschleuningung bewegt.
Diese Beschleuningung wird anhand der letzten drei bekannten Positionen bestimmt 
und anhang dieser Beschleuningung und der aktuellen Geschwindigkeit werden die zukünftigen Positionen abgeleitet.

In \cite{Pfaff2018} werden weitere, szenariospezifische Bewegungsmodelle beschrieben.
Bei dem sogenannten Bias-Corrected Constant Velocity Modell wird das Constant Velocity Modell als Grundlage genommen und ein Korrekturterm eingeführt.
Basierend auf den zuvor beobachteten Schüttgutpartikeln wird ein durchschnittlicher temporaler Bias bestimmt, der -- unter der Annahme, dass der Bias für zukünftige Partikel ähnlich sein wird --
von den zukünftigen Prädiktionen abgezogen wird.
Beim Identical Acceleration Modell wird ebenfalls ein Korrekturterm benutzt, 
der im Gegensatz zum Bias-Corrected Constant Velocity Modell jedoch nicht absolut sondern abhängig von der letzten bekannten Position des Partikels ist.
Auf jedem der zuvor beobachteten Partikeln wird der Wert einer zusätzlichen Beschleuningung bestimmt, die zu einer optimalen Prädiktion führen würde und dann der Durchschnitt dieser Beschleunigungen gebildet.
Dieser wird dann als Korrekturterm auf ein Constant Velocity Modell addiert, sodass sich eine Formel ähnlich zu der des Constant Acceleration Modelles ergibt.
Um das Verhalten von den Partikeln, deren Geschwindigkeit sich der des Förderbands nähert diese aber nicht überschreitet, besser abzubilden als das ein Constant Acceleration Modell tut
wird das Constant Acceleration with Limited Velocity Modell beschrieben. 
Dabei wird die Geschwindigkeit des Förderbands bestimmt und Partikel, die diese erreichen von dann aus mit konstanter Geschwindigkeit weiter prädiziert.
Es wurden zudem zwei Modelle, Constant Acceleration Disallowing Sign Change Modell und Ratio-Based Deceleration Modell, vorgestellt, 
die spezifisch konzipiert wurden um die Bewegungprädiktion orthogonal zur Transportrichtung zu verbessern.

