\chapter{Einleitung}

\begin{itemize}
    \item Maschinelle Lernverfahren sind ein heißes Thema, dass bei vielen existierenden problemen Anwendung findet
    \item Schüttgutsortierung ist ein wichtiges Thema \todo{hier vielleicht dieses 10\% der Energie Beispiel bringen, was irgendwie jeder zu benutzen scheint}
    \item Anwendungsbereiche Schüttgutsortierung
    
\end{itemize}

\todo{Einleitungstext}

\section{Motivation}

\begin{itemize}
    \item State of the Art: große Sortierer
    \item Kooperation ISAS IOSB, \textit{TrackSort} Projekt 
    \item Flächenkamera
    \item 2 geteiltes Problem: Tracking und Prediction
    \item Fokus dieser Arbeit: Prediction
    \item Bewegungsmodelle für verschiedene Schüttgüter von Hand finetunen ist viel Aufwand und schwer
    \item Option: Neuronale Netze einsetzen! 
    \item zwei verschiedene Problemstellungen:
    \item 1. die Position des Teilchens im nächsten Zeitschritt. \textbf{NextStep}
    \item 2. die Position (und die Zeit) die das Teilchen beim Passieren des Düsenarrays haben wird. \textbf{Separator}
\end{itemize}

\todo{arbeit motivieren. Schüttgutsortierung ist ein interessantes Feld, das sich potenziell für ML anbietet.}
\todo{auf jeden fall separator und nextStep Netze unterscheidung erwähnen}

\section{Aufbau der Arbeit}

Das schreibe ich auf, wenn die Gliederung finalisiert ist.

\todo{aufbau Gliederung beschreiben. Ganz am Ende dann, wenn sich nichts mehr ändert}

