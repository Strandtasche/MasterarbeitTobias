

Grundlagen:


- TrackSort Schüttgutsortierung
- Kalman Filter
- NN
	- RNN
	- LSTM

Das Kalman-Filter
Als Kalman-Filter bezeichnet man ein mathematisches Verfahren mit dem Messfehler in realen Messwerten reduziert werden können und nicht messbare Systemgrößen geschätzt werden können. 


[vergangene, aktuelle und zukünftige Systemzustände schätzen]
[Einschränkung Linearität (Extended Kalman) und Gauß rauschen]

Der Zustand des Systems zum Zeitschritt $t$ wird als $y_t$ und die Messung im Zeitschritt $t$ als $z_t$ bezeichnet.

\begin{equation}
	y_t = A y_{t-1} + w, 	w ~ N(0, Q)
\end{equation}

\begin{equation}
	z_t = H y_{t} + v, 	v ~ N(0, R)
\end{equation}

Dabei ist $A$ die Zustandsübergangsmatrix, die den Übergang von einem Zustand in den nächsten beschreibt.
$H$ ist die Messmatrix, die beschreibt wie Messungen aus dem Zustand entstehen und Q und R sind die Kovarianzmatrizen des Systemrauschens beziehungsweise des Messrauschens. 

Das Kalman-Filter funktioniert mittels abwechselnd ausgeführter \textit{predict} und \textit{update} Schritte.

\begin{equation}
\hat{y}'_t = A \hat{y}'_{t-1}
\end{equation}

\begin{equation}
	\hat{P}'_t = A \hat{P}'_{t-1} A^\textit{T} + Q
\end{equation}








Neuronale Netze:

Die Grundsteine des Feldes wurde 1943 von Warren McCulloch und Walter Pitts gelegt, 
die in ihrem Paper ein Neuronenmodell vorschlugen, mit dem sich logische arithmetische Funktionen berechnen lassen. 
Infolge dessen gab verschiedene Forschungsbestrebungen in dem Feld, wie [Examples: TODO!].

[Viele der Begrifflichkeiten, die wir heute noch verwenden wurden 1956 auf der Dartmouth Conference festlegt.]

Nachdem jedoch Marvin Minsky und Seymour Papert zeigten, dass einzelne Perceptrons nicht in der Lage sind linear nicht separierbare Probleme zu lösen sank das Interesse an dem Feld.
