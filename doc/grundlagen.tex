\chapter{Grundlagen}

% Grundlagen:


% - TrackSort Schüttgutsortierung
% - Kalman Filter
% - NN
% 	- RNN
% 	- LSTM

In diesem Kapitel soll eine kurze Einführung in die für das Verständnis der restlichen Arbeit benötigten Themengebiete gegeben werden.
Primär sollen zunächst allgemein neuronale Netze und einige ihrer speziellere Aspekte betrachtet werden 
bevor ein kurzer Blick auf das bei den Experimenten verwendete Schüttgutsortiersystem \textit{TableSort} geworfen wird. 

% hier Könnte man das Kalman Filter Kapitel einfügen
% \section{Das Kalman-Filter}
\todo[inline]{Entscheiden, ob diese Section komplett weg soll}
Als Kalman-Filter bezeichnet man ein mathematisches Verfahren mit dem Messfehler in realen Messwerten reduziert werden können und nicht messbare Systemgrößen geschätzt werden können. 


\todo{vergangene, aktuelle und zukünftige Systemzustände schätzen}
\todo{Einschränkung Linearität (Extended Kalman) und Gauß rauschen}

Der Zustand des Systems zum Zeitschritt \(t\) wird als \(y_t\) und die Messung im Zeitschritt \(t\) als \(z_t\) bezeichnet.

\begin{equation}
	y_t = A y_{t-1} + w, 	w \sim N(0, Q)
\end{equation}

\begin{equation}
	z_t = H y_{t} + v, 	v \sim N(0, R)
\end{equation}

Dabei ist \(A\) die Zustandsübergangsmatrix, die den Übergang von einem Zustand in den nächsten beschreibt.
\(H\) ist die Messmatrix, die beschreibt wie Messungen aus dem Zustand entstehen und Q und R sind die Kovarianzmatrizen des Systemrauschens beziehungsweise des Messrauschens. 

Das Kalman-Filter funktioniert mittels abwechselnd ausgeführter \textit{predict} und \textit{update} Schritte.

\begin{equation}
\hat{y}'_t = A \hat{y}'_{t-1}
\end{equation}

\begin{equation}
	\hat{P}'_t = A \hat{P}'_{t-1} A^\textit{T} + Q
\end{equation}






\section{Neuronale Netze}

Als Neuronale Netze  % beziehungsweise \textit{künstliche neuronale Netze}, wie sie manchmal korrekter genannt werden, 
bezeichnet man in der Informatik Systeme aus künstlichen Neuronen, die heute eine wichtige Rolle im Feld des maschinellem Lernen einnehmen.
Manchmal werden sie korrekter als \textit{künstliche neuronale Netze} bezeichnet um sie von \textit{natürlichen neuronalen Netzen} 
wie dem menschlichen Gehirn zu unterscheiden, nach deren biologischem Vorbild sie modelliert sind.

Die Grundsteine des Feldes wurde bereits 1943 von Warren McCulloch und Walter Pitts gelegt~\cite{mcculloch1943logical}, 
als sie ein Neuronenmodell vorschlugen, mit dem sich logische arithmetische Funktionen berechnen lassen. 
Nach einigen Rückschlägen gab es Phasen von relativ geringer Aufmerksamkeit der wissenschaftlichen Gemeinschaft. 
Erst als um das Jahr 2010 einige herrausragende Ergebnisse, unter anderem im Feld der Sprach- und Bilderkennung, 
mittels neuronaler Netze erzielt wurden, wurde das Interesse an dem Feld neu entfacht. 


% Nachdem jedoch Marvin Minsky und Seymour Papert zeigten, dass einzelne Perzeptrons nicht in der Lage sind linear nicht separierbare Probleme zu lösen sank das Interesse an dem Feld.

\subsection{Perzeptron}
Die kleinste Einheit eines neuronalen Netzes ist das Perzeptron, wie es 1958 von Frank Rosenblatt beschrieben wurde \cite{rosenblatt1958perceptron}.
Es ist eine Art künstliches Neuron, dass eine Reihe an Eingaben entgegen nimmt und einen einzelnen Wert \(o\) ausgibt.

\begin{figure}[h]
    \centering
    % \missingfigure{Grafik Neuron}
	\includegraphics[width=0.5\textwidth]{perceptron-2}
	\caption{Schematischer Aufbau Neuron}
	% \todo{Quelle Bild!}
	\label{fig:singleNeuron}
\end{figure}

Wie in Abbildung~\ref{fig:singleNeuron} zu sehen haben die einzelnen Eingaben \(x_i\) jeweils eine Gewichtung \(w_i\).
Es existiert ein Schwellwert oder \textit{bias}, der normalerweise 
durch eine zusätzliche Eingabe \(x_{m+1}\) mit dem Wert \(+1\) und dem dazugehörigen Gewicht \(w_{m+1}\) modelliert wird.
Den Ausgabewert \(y\) erhält man dadurch, dass man die gewichteten Eingaben aufsummiert und in die Aktivierungsfunktion \( \phi \) des Perzeptrons gibt.
Mathematisch ist die Ausgabe eines Perzeptrons also wie folgt definiert:

\begin{equation}
	y = \phi \Big( \sum_{i= 0}^{m} w_i x_i \Big)
\end{equation}

Ein Überblick über verschiedene Aktivierungsfunktionen, die für solch ein Perzeptron benutzt werden, ist unter~\ref{sec:activationfuncs} zu finden.
Beim Lernen werden die Gewichte der einzelnen Eingaben so angepasst, dass die gewünschte Ausgabe erreicht wird.
\todo{Vielleicht ein paar Sätze zum Perceptron Learning Algorithm?}
Ein einzelnes Perzeptron mit zwei Eingängen kann zur Darstellung der logischen Operatoren AND, OR und NOT genutzt werden

% Letztendlich ist ein solches Perzeptron jedoch nur ein linearer Klassifikator und kann somit 
% zum Beispiel den XOR Operator nicht auflösen.
Ein Perzeptron ist jedoch nur ein linearer Klassifikator und kann dementsprechend zum Beispiel den XOR Operator nicht korrekt abbilden.
Dies zeigten Marvin Minksy und Seymour Papert 1969 in einflussreichen Buch \textit{Perceptrons: an introduction to computational geometry}. \todo{quelle}
\todo{Linear Trennbares / Nicht-linear Trennbares Problem (AND vs XOR)}
Um solche nicht linear-separierbare Probleme zu lösen müssen mehrere Schichten an Neuronen kombiniert werden.

\subsection{Aktivierungsfunktionen}
\label{sec:activationfuncs}
% \todo[inline]{Entscheiden, ob diese Section weg soll, oder auf nur ReLU reduziert werden sollte}
Es gibt verschiedene Aktivierungsfunktionen, die für den Einsatz in neuronalen Netzen in Frage kommen.
Sie sind notwendig, da ohne eine Nicht-Linearität das Netz in eine einfache Regression kollabiert.

Eine Aktivierungsfunktion sollte leicht abzuleiten sein, 
da dies im Rahmen des Trainings mit dem Backpropagation Algorithmus häufig geschieht 
und sonst beträchtlicher Rechenaufwand entsteht.
\todo[inline]{das referenziert dinge von Backprop, die der Leser hier vielleicht noch gar nicht weiß - Nach hinten?}

In der Vergangenheit wurden einige verschiedene Aktivierungsfunktionen verwendet.
% Einige häufig verwendete Aktivierungsfunktionen sollen hier vorgestellt werden.
Jede dieser Funktionen stellt eine Nicht-Linearität dar und nimmt eine einzelne Zahl, wendet eine bestimmte, festgelegte mathematische 
Operation auf diese an und gibt das Ergebnis zurück.
Historisch ist häufig die Sigmoid-Funktion \(\sigma(x)\) verwendet worden, da sie das Verhalten eines natürlichen Neurons gut nachbildet.
\begin{equation}
	\sigma(x) = \frac{1}{1 + e^x} = \frac{e^x}{e^{x + 1}}
	\label{func:Sigmoid}
\end{equation}
In der Praxis jedoch haben sich einige Nachteile der Sigmoid-Funktion gezeigt.
\todo{hier noch genauer drauf eingehen?}
Einige dieser Probleme konnten mit der Verwendung des Tangens hyperbolicus (tanh) behoben werden, 
durchgesetzt haben sich jedoch in letzter Zeit sogenannte \textit{Rectified Linear Units}, oder kurz ReLUs.

% \begin{description}
	
	% hier Könnte man die Absätze zu Sigmoid und TanH einfügen
	% \item[Sigmoid-Funktion] \hfill \\
		\begin{equation}
			f(x) = \frac{1}{1 + e^x} = \frac{e^x}{e^{x + 1}}
			\label{func:Sigmoid}
		\end{equation}
		\begin{equation}
			f'(x) = f(x) * (1 - f(x))
		\end{equation}
		Die mathematische Form der Sigmoid Aktivierungsfunktion ist in Abbildung \ref{sigmoidFunc} zu sehen.
		Sie bildet die reellen Zahlen \(\mathbb{R}\) auf das Intervall \((0,1)\) ab. 
		Für betragsmäßig größer werdende negative Zahlen nähert sich der Rückgabewert \(0\) an,
		ebenso wie für größer werdende positive Zahlen sich der Rückgabewert an \(1\) annähert.

		Die Sigmoid Funktion ist eine historisch häufig genutze Funktion, da sie das Verhalten eines natürlichen Neurons,
		der biologischen Motivation für künstliche Neuronen, gut nachbildet:
		komplette Inaktivität eines Neurons bei Ausgabe 0 bis zum feuern mit maximaler Frequenz bei Ausgabe 1.

		In der Praxis jedoch haben sich einige Nachteile der Sigmoid Funktion gezeigt, weshalb sie quasi nicht mehr genutzt wird.
		Der gewichtigste von diesen ist, dass ihre Ableitung bei großen Beträgen beinah \(0\) ist.
		Dies führt dazu, dass während der Ausführung des Backpropagation-Algorithmus beinah keine Änderungen passieren und dementsprechend das Netz sehr langsam lernt.
		
		\begin{figure}
			\centering
			\includegraphics[width=0.618\textwidth]{Sigmoid}
			\caption{Plot der Sigmoid Funktion}
			\label{sigmoidFunc}
		\end{figure}
		  


	\item[TanH] \hfill \\
		\begin{equation}
			f(x) = \tanh(x) = \frac{e^x - e^{-x}}{e^x + e^{-x}}
		\end{equation}
		\begin{equation}
			f'(x) = 1 - f(x)^2
		\end{equation}
		Die TanH-Aktivierungsfunktion ist in Abbildung \ref{tanhfunction} dargestellt.
		Im Gegensatz zur Sigmoid Funktion bildet sie die reellen Zahlen \(\mathbb{R}\) auf das Intervall \((-1, 1)\) ab.
		Weil sie zentriert um den Nullpunkt ist, wird sie bei realen Anwendungen der Sigmoid Funktion vorgezogen.
		Das Saturationsproblem der Sigmoid Funktion besteht jedoch immer noch.
		\begin{figure}
			\centering
			\includegraphics[width=0.618\textwidth]{Tanh}
			\caption{Plot der Tanh Funktion}
			\label{tanhfunction}
		\end{figure}
	
	% \item[ReLU] \hfill \\
\begin{equation}
	f(x) = \max(0, x)
\end{equation}
\begin{equation*}
	f'(x) = \begin{cases}
	0 &\text{, falls $x < 0$}\\
	1 &\text{, falls $x > 0$}
	\end{cases}
\end{equation*}
	Abbildung~\ref{reluoutput} zeigt das Schaubild einer solchen ReLU. 
	% \todo{ReLU plot neu machen: dicker bei unter 0 und als Vector graphic statt als PNG}
	Die Aktivierung von ReLUs ist ein einfacher Schwellwert, der weit weniger rechenintensiv ist, als die aufwendigen Exponenzialfunktionen von Sigmoid und tanh.
	In der Praxis hat sich zudem gezeigt, dass ReLus deutlich schneller konvergieren als Sigmoid- oder tanh-Neuronen.  
	Krizhevsky et al. haben in ihrem Paper~\cite{NIPS2012_4824} einen Geschwindigkeitsgewinn um Faktor 6 feststellen können.
	Ein Problem, das mit ReLUs jedoch existiert ist, dass einzelne Neuronen während dem Training "absterben" können, falls sie irgendwann in dem Bereich landen, wo der Gradient 0 ist.
	Diese Neuronen sind dann für jeden beliebigen Input inaktiv und können niemals wieder etwas zur Ausgabe des Netzes beitragen.
	Durch die Wahl einer geeigneten Lernrate oder den Einsatz sogenannter Leaky ReLUs lässt sich dies jedoch vermeiden.
	Leaky ReLUs haben im Gegensatz zu normalen ReLUs eine kleine positive Steigung im negativen Bereich.
	\begin{equation}
		f(x) = \begin{cases}
			x &\text{, falls } x  >  0\\
			0.01 x &\text{, falls } x  \leq  0
		\end{cases}
	\end{equation} 
	

	\begin{figure}[h]
		\centering
		\includegraphics[width=0.618\textwidth]{reluPlot.pdf}
		\caption{Schaubild der Ausgabe einer ReLU}
		\label{reluoutput}
	\end{figure}



% \end{description}

% \begin{figure}[h]
%     \centering
%     \begin{subfigure}[t]{0.3\textwidth}
% 		\includegraphics[width=\textwidth]{Sigmoid}
% 		\caption{Sigmoid Funktion}
%     \end{subfigure}
%     \begin{subfigure}[t]{0.3\textwidth}
% 		\includegraphics[width=\textwidth]{Tanh}
% 		\caption{TanH Funktion}
%     \end{subfigure}
%     \begin{subfigure}[t]{0.3\textwidth}
%         \includegraphics[width=\textwidth]{ReLu}
%         \caption{ReLU}
%     \end{subfigure}
%     \caption{Häufig verwendete Aktivierungsfunktionen}
%     \label{eval:function}
% \end{figure}



\subsection{Feedforward Netze}

% \textbf{absatz über Feedforward Netze. Basic}
\color{blue}
\begin{itemize}
	\item Definition (keine Kreise oder Schleifen). (Kommentar Florian: \"Rückkopplung\")
	\item Gegenstück zum RNN
	\item grundlegende Architektur in Layern 
	\item Aktivierungsfunktionen in Layern
	\item Outputlayer: Verschiedene Aktivierungsfunktionen:
	\item Linear für regression, z.B. Softmax für Wahrscheinlichkeitsverteilung Softmax
\end{itemize}
\color{black}



Als Feedforward Netz bezeichnet man ein neuronales Netz, zwischen dessen Knoten keine Kreise oder Schleifen existieren.
Ein Netz, in dem es solche Verbindungen gibt, bezeichnet man als \textit{Rekurrentes Neuronales Netz}.
Die Informationen wandert in der Verarbeitungsrichtung von den Eingabeneuronen zu den Ausgabeneuronen.
Für gewöhnlich sind die einzelnen Knoten in Schichten, sogenannten Layern, organisiert.
Auf ein Eingabe- oder auch Input Layer folgen eine Menge an sogenannten Hidden Layers.
Den Schlussstein bildet das Ausgabe- oder auch Output Layer.
Die Neuronen eines einzelnen Hidden Layers sind meist uniform und verwenden identische Aktivierungsfunktionen, wie sie in Abschnitt~\ref{sec:activationfuncs} beschrieben sind.
Je nachdem welche Aufgabe das neuronale Netz erfüllen soll können unterschiedliche Aktivierungsfunktionen für das Output Layer sinn ergeben.
Häufig genutzt ist die \textit{Softmax} Funktion für Klassifikation, da sie eine Wahrscheinlichkeitsverteilung als Ausgabe hat.
Für Regression benutzt man lineare Aktivierungsfunktionen für das Output Layer. 
Durch die Hintereinanderschaltung von mehreren Layern von Neuronen können auch Probleme, die nicht linear-separierbar sind, gelöst werden.
Tatsächlich sind solche Neuronalen Netze bewiesenermaßen universale Funktionsapproximatoren - 
sie können mit endlich vielen Neuronen in den Hidden Layers beliebige kontinuierliche Funktionen auf kompakten Subsets von \(\mathbb{R}^n\) approximieren.
\todo[inline]{mit Quelle belegen? Universal approximation theorem }
Es gibt viele verschiedene Unterkategorien von Feedforward Netzen, die in verschiedensten Bereichen Verwendung finden.
Convolutional Neural Networks zeichnen sich durch eine spezielle Struktur aus sogenannten Convolutional Layers aus, die als Featureextractor fungieren.
Diese Art von Netz findet unter anderem in der Bildverarbeitung Anwendung.

\subsection{Backpropagation}

\todo[inline]{Optional: am Ende entscheiden ob ich noch mehr Kontent möchte und dann gegebenenfalls ausformulieren}
Der Backpropagation Algorithmus ist ein Verfahren, mit denen künstliche neuronale Netze in der Lage sind, komplizierte Zielfunktionen einzulernen.
Es ist eine Methode, bei der effizient der Gradient der Fehlerfunktion in Abhängigkeit vom Gewicht der einzelnen Kanten im Netz bestimmt werden kann,
was dann für einen Gradientenabstieg verwendet werden kann. 

\color{blue}
\begin{itemize}
	\item Definition und Beschreibung
	\item Nur supervised learning: Gradient der Fehlerfunktion wird benötigt \(\rightarrow\) Tatsächliches Ergebnis muss bekannt sein.
	\item "Finden einer Funktion, die am besten die Inputs auf die outputs mapt"
\end{itemize}
\color{black}

\subsection{Performance-Maß}

Neuronale Netze werden in der Regel danach bewertet, wie gut sie mit unbekannte, das heißt nicht im Training vorgekommenen, Daten umgehen können.
% Das essenzielle Maß nachdenen neuronale Netze bewertet werden, ist wie gut sie mit neuen, unbekannten Daten umgehen, die nicht in den Trainingsdaten vorhanden waren. 
Diese Eigenschaft von Trainingsdaten auf unabhängige Testdaten zu schließen wird Generalisierung genannt. 

Als Overfitting bezeichnet man es, wenn ein Modell sich zu sehr an ein gegebenes Datenset anpasst und 
dafür in Kauf nimmt zusätzliche oder zukünftige Daten schlechter zu repräsentieren.
Das System wird also schlechter darin zu generalisieren.
Im Feld des überwachten Lernens beziehungsweise der neuronalen Netze ist Overfitting daran zu erkennen,
dass die Qualität die Ausgaben des Netzes auf dem Trainingsdatenset sich weiter verbessert,
während sie auf dem Testdatenset schlechter wird.
Dies kann zum Beispiel der Fall sein, wenn das Modell Rauschen in den Testdaten als Teil der zugrundeliegenden Struktur interpretiert. 


Dem Overfitting gegenüber steht das Underfitting. 
Als Underfitting bezeichnet man wenn das Netz nicht in der Lage ist
eine ausreichend gute Performance auf den Trainingsdaten zu erreichen.
Das kann passieren wenn das Modell nicht ausreichend komplex ist um die zugrundeliegende Struktur der Daten abzubilden.


Sowohl Overfitting als auch Underfitting hängen mit der Kapazität eines Netzes zusammen.
Ist die Kapazität zu gering, kann es sein, dass das Netz daran scheitert die Trainingsdaten zu lernen.
Ist die Kapazität zu groß, so kann es passieren, dass das Netz, umgangsprachlich ausgedrückt, die Trainingsdaten einfach auswendig lernt.
\todo{umgangsprachlich mit einer besseren Formulierung ersetzen...}

Dies ist beispielhaft in Abbildung~\ref{fig:capacity} zu sehen.
Aus der zugrundeliegenden \( \cos \)-Funktion werden Stichproben mit einem Rauschterm entnommen. 
Stellvertretend für das Lernen mit neuronalen Netzen wird hier lineare Regression benutzt,
um die Parameter eines Polynoms zu lernen.
Die Kapazität des Modells wird hier durch den Grad des Polynoms festgelegt.

In Abbildung~\ref{subfig:underfitting} wird ein Polynom mit dem Grad 1 gelernt.
Die Kapazität ist zu niedrig und dementsprechend schafft das Modell es nicht die Stichproben akkurat zu repräsentieren.
In Abbildung~\ref{subfig:rightfitting} ist der Grad des Polynoms 4.
Das resultierende Modell ist eine gute Approximation der ursprünglichen Funktion.
In Abbildung~\ref{subfig:overfitting} ist der Grad des Polynoms mit 17 deutlich zu hoch.
Man kann am resultierenden Modell erkennen wie es dem Rauschen der Samples folgt und wie es an den Rändern des betrachteten Bereichs weit ausschlägt.

Die beste Methode um Overfitting zu vermeiden ist es mehr Trainingsdaten zu bekommen.
In der Praxis kann dies oft umständlich, kostspielig oder sogar unmöglich sein, 
weshalb auf das Generiern von synthetischen Trainingsdaten, auch Datenaugmentierung genannt, zurückgegriffen wird.
% There is no Data like more Data
Auch der Einsatz von Regularisierungsverfahren kann helfen Overfitting zu verringern.
Als letzter Schritt kann die Kapazität der Architektur gesenkt werden,
zum Beispiel indem die Anzahl der zur Verfügung stehenden Layer reduziert wird.

\begin{figure}[h]
    \centering
	
	\begin{subfigure}[t]{0.6\textwidth}
		\includegraphics[width=\textwidth]{plotUnderfitting.pdf}
		\caption{Underfitting}
		\label{subfig:underfitting}
	\end{subfigure}
	% \quad
	\begin{subfigure}[t]{0.6\textwidth}
		\includegraphics[width=\textwidth]{plotPassendesModell.pdf}
		\caption{Passendes Modell}
		\label{subfig:rightfitting}
	\end{subfigure}
	% \quad
	\begin{subfigure}[t]{0.6\textwidth}
        \includegraphics[width=\textwidth]{plotOverfitting.pdf}
		\caption{Overfitting}
		\label{subfig:overfitting}
	\end{subfigure}
	\caption{Visualisierung von unterschiedliche Kapazitäten}
	\label{fig:capacity}
\end{figure}

\color{blue}
\begin{itemize}
	\item Methoden um Overfitting zu vermeiden:
	\item Mehr Trainingsdaten (z.\,B. durch Data-Augmentation)
	\item Regularisierung (L1, \(L_2\) (siehe unten), dropout)
\end{itemize}
\color{black}

\subsection{Regularisierung} \label{ssec:Regul}

\color{blue}
\begin{itemize}
	\item Regularisierung Definition
	\item Mathematische Formel Darstellung
	\item \(L_1\) und \(L_2\) Regularisierung
	\item L0 - warum nicht?
	\item (Maybe Dropout)
	\item Early Stopping? (das internet sagt, es ist eine art von Regularisierung und falls ich sie verwenden sollte, sollte ich sie hier erwähnen)
\end{itemize}
\color{black}

Als Regularisierung bezeichnet man eine Technik, die zur Vermeidung von Overfitting verwendet wird.
Sie wird in der Hoffnung angewendet, dass das Modell mit Regularisierung besser generalisiert als ohne.

Eine Möglichkeit ist, dass zur Loss Funktion ein Regularisierungsterm \(R\) hinzugefügt wird, 
der die Kosten basierend auf der Komplexität des Systems erhöht.

\begin{equation}
	\min_f \sum\limits_{i=1}^{m} V(f(\vec{x}_i), \vec{y}_i) + \lambda R(f)
\end{equation} 

Dabei ist \(V\) die Loss Funktion, beispielsweise \textit{Mean-Square-Error} oder \textit{Mean-Absolute-Error}.
\(n\) ist die Anzahl der Feature-Label-Paare,
\(x_i\) und \(y_i\) sind die einzelnen Eingabefeatures und das dazugehörige Label.
Die Funktion \(f\) ist in unserem Fall das neuronale Netz, das die Features entgegen nimmt.
\(\lambda\) ist ein Parameter, der die Gewichtung des Regularisierungsterm festlegt.
Wählt man diesen Parameter zu klein, so kann es sein, dass das Modell trotz Regularisierung noch immer overfittet.
Wählt man ihn zu groß, so kann es sein, dass das Modell das Problem nicht mehr korrekt abbildet und es zu Underfitting kommt.
Der Regularisierungsterm \(R\) wird so gewählt, dass er die Komplexität der Funktion \(f\) widerspiegelt.
Ein gutes Maß für die Komplexität eines neuronalen Netzes sind die Gewichte zwischen den Neuronen.
Beispiele für \(R\) wären zum Beispiel die \(L_1\)- oder die \(L_2\)-Regularisierung. % die jeweils mit der \(L_1\) beziehungsweise mit der \(L_2\)-Norm arbeiten.
Der entscheidende Unterschied zwischen den beiden ist der unterschiedliche Strafterm, zu sehen in~\ref{eqn:MSE-L1} für \(L_1\) und~\ref{eqn:MSE-L2} für \(L_2\). 
Die Fehlerfunktionen sind jeweils MSE mit dazugehörigen Strafterm.

\begin{equation} \label{eqn:MSE-L1}
	J(X, Y) = \frac{1}{m} \sum_{i=1}^{m} (\vec{y}^{(i)} - \hat{\vec{y}}^{(i)})^2 + \sum_{j, k} (|\mat{W}_{j,k}|)
\end{equation} 

\begin{equation} \label{eqn:MSE-L2}
	J(X, Y) = \frac{1}{m} \sum_{i=1}^{m} (\vec{y}^{(i)} - \hat{\vec{y}}^{(i)})^2 + \sum_{j, k} (\mat{W}_{j,k}^2)
\end{equation} 


Ein Regressionsmodell, das \(L_1\)-Regularisierung verwendet wird auch als Lasso Regression bezeichnet, 
während ein Modell mit \(L_2\)-Regularisierung als Ridge Regression beschrieben werden kann.
Vergleicht man die beiden Ansätze, so schrumpft die \(L_1\)-Norm weniger wichtige Gewichte auf 0, was zu dünn besetzten Gewichtsvektoren führt.
Dies kann eine wünschenswerte Eigenschaft sein.
Im Gegensatz dazu hat die \(L_2\)-Regularisierung, den Vorteil, dass sie effizienter berechnen kann.
Der Strafterm von \(L_2\) hat eine geschlossene Form und kann in Form einer Matrix angewendet werden, während die Funktion von \(L_1\) auf Grund des Betrags eine nicht-differenzierbar ist.


\todo{absatz zu (L0 und warum man es nicht benutzt?)}

Eine weitere Regularisierungstechnik ist Dropout~\cite{JMLR:v15:srivastava14a}.
Dabei werden während dem Training eines neuronalen Netzes 
mit einer festgelegten Wahrscheinlichkeit zufällig Neuronen und die dazugehörigen Verbindungen abgeschaltet, 
wie in Abbildung~\ref{fig:dropout} dargestellt.
\todo{Ist die Grafik essenziell fürs Verständnis?}
Dies soll insofern Overfitting vermeiden, dass es übermässige Koadaption von mehreren Neuronen erschwert.
Dropout als Technik wird insbesondere bei tiefen neuronalen Netzen mit einer hohen Anzahl von Hidden Layern eingesetzt. 

\begin{figure}[h]
    \centering
    \begin{subfigure}[t]{0.4\textwidth}
		\includegraphics[width=\textwidth]{neuralNet}
		\caption{Unverändertes neuronales Netz}
    \end{subfigure}
    \begin{subfigure}[t]{0.4\textwidth}
		\includegraphics[width=\textwidth]{neuralNet_dropped}
		\caption{Nach Dropout Anwendungen.}
	\end{subfigure}
    \caption{Beispiel Dropout-Regularisierung~\cite{JMLR:v15:srivastava14a}}
    \label{fig:dropout}
\end{figure}


\section{TableSort System}

\todo[inline]{Viel stuff über das TableSort System}
\color{blue}

\begin{itemize}
	\item kleiner, experimenteller Bandsortierer~\cite{doll2015}
	\item Entstanden in Kooperation zwischen dem Fraunhofer IOSB, Abteilung Sichtprüfsysteme, und dem Institut für Intelligente Sensor Aktor Systeme des Karlsruher Institut für Technologie.
	\item Im Rahmen des \textit{TrackSort} Projekts
	\item Gedacht für Experimente, wenn es zu aufwendig ist das mit dem großen großen zu machen und zum Mitnehmen auf Messen.
	\item 2 Modi: mit Förderband und mit Rutsche
	\item Mit Flächenkamera für TrackSort als auch die Zeilenkamera sind dargestellt.
	\item Ringlicht (Refence später)
	\item Die Zeilenkamera wird zurzeit in industriellen Schüttgutsortieranlagen verwendet, ist aber nicht optimal (Siehe all die Literatur)
\end{itemize}
\color{black}


Der \textit{TableSort} Schüttgutsortierer ist eine modulare Versuchsplattform, die konzipiert wurde um neue Schüttgutsortierkonzepte in einem kleineren Rahmen experimentell erproben zu können.
Industrielle Schüttgutsortieranlagen sind sehr groß und um etwas an ihrer Konfiguration zu ändern ist sowohl zeitaufwendig als auch arbeitsintensiv, 
weshalb es sinnvoll ist, eine Plattform zu haben, die flexibel und transportabel ist und umgebaut werden kann.
Das System ist im Rahmen des \textit{TrackSort} Projekts in Kooperation zwischen dem Fraunhofer IOSB, Abteilung Sichtprüfsysteme, und dem Institut für Intelligente Sensor Aktor Systeme des Karlsruher Institut für Technologie entstanden\cite{doll2015}.
Es kann in zwei verschiedenen Konfigurationen bezüglich des Schüttguttransports benutzt werden.
Einmal werden die Schüttgutpartikel mittels eines Förderbands beschleunigt - diese Konfiguration ist in Abbildung~\ref{fig:tablesortsystem} zu sehen.
Dazu kommt die Möglichkeit dieses Förderband mit einer Rutsche zu ersetzen, sodass die Schüttgutpartikel stattdessen durch die Schwerkraft beschleunigt werden. 
Im Rahmen dieser Arbeit wurden die Flächenkamera Konfiguration mit einem Ringlicht zur Beleuchtung verwendet.
Schematisch ist der Ablauf der Schüttgutsortierung in Abbildung~\ref{fig:aufbau_tablesort} zu sehen.
\begin{figure}[h]
	% \missingfigure{Bild von TablesortSystem}
	\includegraphics[width=\textwidth]{TrackSortPic}
	\caption{TableSort Schüttgutsortiersystem \cite{fraunhoferiosb2017}}
	% \todo{Quelle Bild!}
	\label{fig:tablesortsystem}
\end{figure}


\begin{figure}[h]
    \centering
    \def\svgwidth{\columnwidth}
	\input{img/Aufbau-moved.pdf_tex}
	\caption[Schematische Darstellung des optischen Bandsortierers TableSort nach~\cite{Pfaff2017}.]{
		Schematische Darstellung des optischen Bandsortierers TableSort nach~\cite{Pfaff2017}.
	}
	\label{fig:aufbau_tablesort}

\end{figure}


