
\documentclass[german]{isasthesis}

\usepackage[utf8]{inputenc}
\usepackage[ngerman]{babel}

\usepackage{graphicx}
\usepackage{csquotes}
\usepackage{adjustbox}
\usepackage{textcomp}
\usepackage{booktabs}
\usepackage{makecell}
\usepackage{enumitem}
\usepackage{float}
\usepackage{adjustbox}
% \usepackage[margin=1in]{geometry} % choose margins to suit your needs

\usepackage{csvsimple}
\MakeOuterQuote{"}

% \usepackage{caption}
\usepackage{subcaption}

\graphicspath{ {img/} }

\usepackage{listings}
\usepackage{color}
\usepackage{courier}
% \usepackage{siunitx}
\usepackage[binary-units]{siunitx}
\DeclareSIUnit\px{px}

\usepackage{xcolor}

\usepackage{pgf}
% \usepackage{import}
\usepackage{todonotes}
\usepackage{microtype}

\usepackage[style=ieee-alphabetic, sorting=none, backend=bibtex]{biblatex}

\addbibresource{sources.bib}

\usepackage{hyperref}


\hypersetup{pdfauthor={Tobias Hornberger},
            pdftitle={Bewegungsmodelle in der Schüttgutsortierung mittels ML},
            pdfsubject={Thesis},
            pdfkeywords={Neuronale Netze, Schüttgutsortierung, Tensorflow, TableSort}}

%%%%%%%%%%%%%%%%%%%%%%%%%%%%%%%%%%
% Edit notation.tex if necessary
%%%%%%%%%%%%%%%%%%%%%%%%%%%%%%%%%%

\colorlet{punct}{red!60!black}
\definecolor{background}{HTML}{EEEEEE}
\definecolor{delim}{RGB}{20,105,176}
\colorlet{numb}{magenta!60!black}

\lstdefinelanguage{json}{
    basicstyle=\normalfont\ttfamily\small,
    numbers=left,
    numberstyle=\scriptsize,
    stepnumber=1,
    numbersep=8pt,
    showstringspaces=false,
    breaklines=true,
    frame=lines,
    backgroundcolor=\color{background},
    literate=
     *{0}{{{\color{numb}0}}}{1}
      {1}{{{\color{numb}1}}}{1}
      {2}{{{\color{numb}2}}}{1}
      {3}{{{\color{numb}3}}}{1}
      {4}{{{\color{numb}4}}}{1}
      {5}{{{\color{numb}5}}}{1}
      {6}{{{\color{numb}6}}}{1}
      {7}{{{\color{numb}7}}}{1}
      {8}{{{\color{numb}8}}}{1}
      {9}{{{\color{numb}9}}}{1}
      {:}{{{\color{punct}{:}}}}{1}
      {,}{{{\color{punct}{,}}}}{1}
      {\{}{{{\color{delim}{\{}}}}{1}
      {\}}{{{\color{delim}{\}}}}}{1}
      {[}{{{\color{delim}{[}}}}{1}
      {]}{{{\color{delim}{]}}}}{1},
}


%%%%%%%%%%%%%%%%%%%%%%%%
% Document properties
%%%%%%%%%%%%%%%%%%%%%%%%

\title{Ableitung von Bewegungsmodellen für Anwendungen in der Schüttgutsortierung mittels Machine Learning}
\author{Tobias Hornberger}
\date{31. Dezember 2018}

\thesistype{Masterarbeit}
\discussant{Prof. Dr.-Ing.  Thomas  Längle}
\firstsupervisor{Dipl.-Inform. Florian Pfaff}
\secondsupervisor{Georg Maier, M.\,Sc.}
\thirdsupervisor{Dr.-Ing. Benjamin Noack}

%%%%%%%%%%%%%%%%%%%%%%%%
% Document
%%%%%%%%%%%%%%%%%%%%%%%%

\begin{document}
    \maketitle

    \begin{abstract}
        This work was a cooperation with the Fraunhofer IOSB and centered around their TableSort bulk material sorting system.
        The aim was to provide an alternative to the labour-intensive task of fine tuning motion models for particle tracking by hand. 
        This was achieved by using neural networks, which were implemented using the Tensorflow framework. 
    \end{abstract}

    \maketoc

    \chapter{Einleitung}

\begin{itemize}
    \item Schüttgutsortierung ist ein wichtiges Thema \todo{hier vielleicht dieses 10\% der Energie Beispiel bringen?}
    \item Anwendungsbereiche Schüttgutsortierung
    \item Maschinelle Lernverfahren sind ein heißes Thema, dass bei vielen existierenden problemen Anwendung findet
\end{itemize}

Schüttgüter und ihr Transport sind aus unserer modernen, globalisierten Welt nicht mehr wegzudenken.
Ob es sich um Lebensmittel wie Getreide, Kaffeebohnen oder Zucker, Bergbauerzeugnisse wie Eisenerze oder Kohle, oder [Granulat/Pellets] handelt, 
[...]
[nicht-destruktiv]
In dieser Arbeit soll es insbesondere um die bilddatenbasierte beziehungsweise optische Schüttgutsortierung gehen.



\todo{Einleitungstext}

\section{Motivation}

\color{blue}
\begin{itemize}
    \item State of the Art: große Sortierer 
    \item Kooperation ISAS IOSB, \textit{TrackSort} Projekt 
    \item Flächenkamera
    \item 2 geteiltes Problem: Tracking und Prediction
    \item Fokus dieser Arbeit: Prediction
    \item Bewegungsmodelle für verschiedene Schüttgüter von Hand finetunen ist viel Aufwand und schwer
    \item Option: Neuronale Netze einsetzen! 
    \item zwei verschiedene Problemstellungen:
    \item 1. die Position des Teilchens im nächsten Zeitschritt. \textbf{NextStep} für Trackingphase
    \item 2. die Position (und die Zeit) die das Teilchen beim Passieren des Düsenarrays haben wird. \textbf{Separator}
    \item Aktuell: auf Messungen - kein Vollständiger Schätzer
\end{itemize}
\color{black}

\todo[inline]{arbeit motivieren. Schüttgutsortierung ist ein interessantes Feld, das sich potenziell für ML anbietet.}
\todo[inline]{auf jeden fall separator- und NextStep-Netze unterscheidung erwähnen}

Der Großteil der heute in der Industrie eingesetzten optischen Schüttgutsortierer verwenden Zeilenkameras.
Dabei muss jedoch die Annahme getroffen werden, dass die Schüttgutpartikel keinerlei Geschwindigkeit orthogonal zur Transportrichtung haben,
weil diese nicht erfasst werden kann.
In Abbildung~\ref{fig:predMissed} ist dargestellt wie es zu einer Fehlseparierung kommen kann, wenn diese Annahme verletzt wird.
Durch den Einsatz von Flächenkameras ist es möglich, die Position eines Partikels auf dem Förderband zu mehreren Zeitpunkten zu bestimmen.
Basierend auf diesen Informationen sollen die Trajektorien der einzelnen Partikel vorhergesagt werden.
Diese sollen dazu verwendet werden, um die Sortierqualität zu steigern. 
Im Rahmen des \textit{TrackSort} Projekts, 
einer Kooperation zwischen dem Lehrstuhls für Intelligente Sensor-Aktor-Systeme (ISAS) des Karlsruher Instituts für Technologie
und dem Fraunhofer-Institut für Optronik, Systemtechnik und Bildauswertung (IOSB), 
wurde die Verbesserung der Schüttgutsortierung durch den Einsatz von Trackingverfahren betrachtet.
\todo[inline]{passt das hier rein, oder soll ich das eher wo anders erwähnen?}

\begin{figure}[h]
    \centering
    \includegraphics[width=0.8\textwidth]{PredictionMissed_translated.pdf}
    \caption{Darstellung einer Fehlseparierung durch die Annahme, dass es keine Bewegung orthogonal zur Transportrichtung gibt. 
    Übersetzt aus~\cite{Pfaff2018}.}
    \label{fig:predMissed}
\end{figure}


Um die zukünftigen Trajektorien der Partikel aus den vergangenen Positionen vorherzusagen, existieren verschiedene Bewegungsmodelle.
Diese liefern für unterschiedliche Situationen, mit unterschiedlichen Bandgeschwindigkeiten, Schüttguttypen oder Prädiktionsdistanzen, unterschiedlich gute Ergebnisse.  
Mittels neuronaler Netze können komplexe, nichtlineare Muster aus bestehenden Daten gelernt werden, ohne dass explizit auf diese hingewiesen werden müssen.
Es ist also denkbar, dass sie in der Lage wären die Bewegungen des Schüttguts zu lernen.
Die für den Einsatz von neuronalen Netzen essenziellen Daten in ausreichender Menge zu sammeln ist definitiv möglich. 
In dieser Arbeit soll nun erforscht werden, inwiefern der Einsatz von neuronalen Netzen zu einer Verbesserung gegenüber den bestehenden Ergebnissen führen kann.
Dabei wird exemplarisch mit Daten von dem \textit{TableSort} Schüttgutsortier gearbeitet.

Dafür sollen im Rahmen dieser Arbeit zwei verschiedene Prädiktionsprobleme durch neuronale Netze gelöst werden.
Einerseits soll vorhergesagt werden, an welcher Position sich ein Teilchen im nächsten Zeitschritt befinden wird.
Solche Netze werden von hier an als NextStep-Netze bezeichnet.
Eine Visualisierung dieser Aufgabe ist in Abbildung~\ref{fig:visualsNextstep} zu sehen.
Dies hilft dabei, das Zuordnungsproblem bei Multi-Target-Tracking zu lösen.
Andererseits soll vorhergesagt werden, an welcher Position und wann ein Teilchen das Druckluftdüsenarray passieren wird.
Diese Problemstellung wird von sogenannten Separator-Netzen gelöst.
Eine Visualisierung dieser Aufgabe ist in Abbildung~\ref{fig:visualsSeparator} zu sehen.
Die Qualität dieser Prädiktion ist ausschlaggebend für den Erfolg der Separation.


\begin{figure}[p]
    \centering
    \includegraphics[width=0.8\textwidth]{NextStep-45kDecaySteps-ZylinderReal_2018-11-29.pdf}
    \caption[Visualisierung einer gelösten Probleminstanz eines NextStep-Netzes]{Visualisierung einer gelösten 
    Probleminstanz eines NextStep-Netzes. Die Transportrichtung ist entlang der \(y\)-Achse nach oben.
    Die Features sind die zeitlich aufeinanderfolgenden, beobachteten Positionen eines Partikels.
    Das Label ist die Position des entsprechenden Partikels im nächsten Zeitschritt.
    }
    \label{fig:visualsNextstep}
\end{figure}


\begin{figure}[p]
    \centering
    % \missingfigure{Real_Weizen_final_separatorExample.pdf}
	\includegraphics[width=0.8\textwidth]{Real_Weizen_final_separatorExample.pdf}
    \caption[Visualisierung einer gelösten Probleminstanz eines Separator-Netzes]{Visualisierung einer gelösten Probleminstanz 
    eines Separator-Netzes.
    Wie in Abbildung~\ref{fig:visualsNextstep} ist die Transportrichtung entlang der \(y\)-Achse nach oben.
    Die Features sind erneut die zeitlich aufeinanderfolgenden, beobachteten Positionen eines Partikels.
    Das eine Element des Label ist die Position entlang der \(x\)-Achse an der das entsprechende Partikel das Druckluftdüsenarray passiert.
    Das zweite Element des Labels -- der Zeitpunkt an dem es das Druckluftdüsenarray passiert -- ist nicht abgebildet.
    }
	
	\label{fig:visualsSeparator}
\end{figure}



\section{Aufbau der Arbeit}

Das schreibe ich auf, wenn die Gliederung finalisiert ist.

\todo{aufbau Gliederung beschreiben. Ganz am Ende dann, wenn sich nichts mehr ändert}


    \chapter{Grundlagen}
\label{cap:basics}
% Grundlagen:


% - TrackSort Schüttgutsortierung
% - Kalman Filter
% - NN
% 	- RNN
% 	- LSTM

In diesem Kapitel wird eine kurze Einführung in die für das Verständnis der restlichen Arbeit benötigten Themengebiete gegeben.
Primär werden zunächst allgemein neuronale Netze und einige ihrer spezielleren Aspekte betrachtet, 
bevor ein kurzer Blick auf das bei den Experimenten verwendete Schüttgutsortiersystem \textit{TableSort} geworfen wird. 
Am Ende des Kapitels wird auf den aktuellen Stand der Technik eingegangen.
%  wie das Problem der Prädiktion bei optischen Schüttgutsorierern vor der Arbeit gelöst wird.
Die Ausführungen zu den neuronalen Netzen sind, wo nicht anders angemerkt, basierend auf \cite{Goodfellow-et-al-2016}, \cite{Murphy:2012:MLP:2380985} und \cite{nielsen2015neural}.
% \todo[inline]{Primäre Quellen sind hier \cite{Goodfellow-et-al-2016}, \cite{Murphy:2012:MLP:2380985} und \cite{nielsen2015neural} aber ich bin mir nicht sicher wo ich das richtig vermerke.}

% hier Könnte man das Kalman Filter Kapitel einfügen
% \section{Das Kalman-Filter}
\todo[inline]{Entscheiden, ob diese Section komplett weg soll}
Als Kalman-Filter bezeichnet man ein mathematisches Verfahren mit dem Messfehler in realen Messwerten reduziert werden können und nicht messbare Systemgrößen geschätzt werden können. 


\todo{vergangene, aktuelle und zukünftige Systemzustände schätzen}
\todo{Einschränkung Linearität (Extended Kalman) und Gauß rauschen}

Der Zustand des Systems zum Zeitschritt \(t\) wird als \(y_t\) und die Messung im Zeitschritt \(t\) als \(z_t\) bezeichnet.

\begin{equation}
	y_t = A y_{t-1} + w, 	w \sim N(0, Q)
\end{equation}

\begin{equation}
	z_t = H y_{t} + v, 	v \sim N(0, R)
\end{equation}

Dabei ist \(A\) die Zustandsübergangsmatrix, die den Übergang von einem Zustand in den nächsten beschreibt.
\(H\) ist die Messmatrix, die beschreibt wie Messungen aus dem Zustand entstehen und Q und R sind die Kovarianzmatrizen des Systemrauschens beziehungsweise des Messrauschens. 

Das Kalman-Filter funktioniert mittels abwechselnd ausgeführter \textit{predict} und \textit{update} Schritte.

\begin{equation}
\hat{y}'_t = A \hat{y}'_{t-1}
\end{equation}

\begin{equation}
	\hat{P}'_t = A \hat{P}'_{t-1} A^\textit{T} + Q
\end{equation}






\section{Neuronale Netze}

Als neuronale Netze  % beziehungsweise \textit{künstliche neuronale Netze}, wie sie manchmal korrekter genannt werden, 
bezeichnet man in der Informatik Systeme aus künstlichen Neuronen, die heute eine wichtige Rolle im Feld des maschinellen Lernens einnehmen.
Manchmal werden sie korrekter als \textit{künstliche neuronale Netze} bezeichnet, um sie von \textit{natürlichen neuronalen Netzen} 
wie dem menschlichen Gehirn zu unterscheiden, nach deren biologischem Vorbild sie modelliert sind.
Neuronale Netze sind ein mächtiges Werkzeug. 
Aufgrund ihrer Fähigkeit komplexe, nichtlineare Muster und Zusammenhänge in Daten ohne explizite Vorgaben zu lernen, 
% ohne dass sie explizit darauf programmiert werden müssen diese zu finden,
eröffnen sie in vielen Forschungsrichtungen neue Möglichkeiten.

% \todo[inline]{Motivation, warum neuronale Netze gut sind - Patternrecognition die zu kompliziert sind sie von hand zu implementieren.}

Die Grundsteine des Feldes wurden bereits 1943 von Warren McCulloch und Walter Pitts gelegt~\cite{mcculloch1943logical}, 
als sie ein Neuronenmodell vorschlugen, mit dem sich logische arithmetische Funktionen berechnen lassen. 
Zwischenzeitlich gab es einige Phasen von relativ geringer Aufmerksamkeit der wissenschaftlichen Gemeinschaft. 
Erst als um das Jahr 2010 einige herausragende Ergebnisse, unter anderem im Feld der Sprach- und Bilderkennung, 
mittels neuronaler Netze erzielt wurden, wurde das Interesse an dem Feld neu entfacht. 


% Nachdem jedoch Marvin Minsky und Seymour Papert zeigten, dass einzelne Perzeptrons nicht in der Lage sind linear nicht separierbare Probleme zu lösen sank das Interesse an dem Feld.

\subsection{Perzeptron}
Die kleinste Einheit eines neuronalen Netzes ist das Perzeptron, wie es 1958 von Frank Rosenblatt in \cite{rosenblatt1958perceptron} beschrieben wurde.
Es ist ein künstliches Neuron, das eine Reihe an Eingaben entgegen nimmt und einen einzelnen Wert \(y\) ausgibt.

\begin{figure}[h]
    \centering
    % \missingfigure{Grafik Neuron}
	\includegraphics[width=0.5\textwidth]{perceptron-2}
	\caption[Schematischer Aufbau eines Perzeptrons]{Aufbau eines Perzeptrons: Die Eingaben $x_i$ werden gewichtet mit $w_i$ aufsummiert und nach Anwendung der Aktivierungsfunktion $\phi$ als $y$ ausgegeben.}
	% \todo{Quelle Bild!}
	\label{fig:singleNeuron}
\end{figure}

Wie in Abbildung~\ref{fig:singleNeuron} zu sehen ist, haben die einzelnen Eingaben \(x_i\) jeweils eine Gewichtung \(w_i\).
Es existiert ein Schwellwert oder \textit{bias}, der normalerweise 
durch eine zusätzliche Eingabe \(x_{m+1}\) mit dem Eingabewert \(+1\) und dem dazugehörigen Gewicht \(w_{m+1}\) modelliert wird.
Den Ausgabewert \(y\) erhält man dadurch, dass man die gewichteten Eingaben aufsummiert und in die Aktivierungsfunktion \( \phi \) des Perzeptrons gibt.
Mathematisch ist die Ausgabe eines Perzeptrons somit wie folgt definiert:

\begin{equation}
	y = \phi \Big( \sum_{i= 0}^{m} w_i x_i \Big) \, .
\end{equation}

Mehr Details über verschiedene Aktivierungsfunktionen, die für solch ein Perzeptron benutzt werden können, ist in Unterabschnitt~\ref{sec:activationfuncs} zu finden.
Beim Lernen werden die Gewichte \(w_i\) so angepasst, dass die gewünschte Ausgabe für die jeweilige Eingabe erreicht wird.
Bei der Auswertung verändern sich die Gewichte nicht mehr.
% \todo{Vielleicht ein paar Sätze zum Perceptron Learning Algorithm?}
Ein einzelnes Perzeptron mit zwei Eingängen kann zur Darstellung der logischen Operatoren AND, OR und NOT genutzt werden.

% Letztendlich ist ein solches Perzeptron jedoch nur ein linearer Klassifikator und kann somit 
% zum Beispiel den XOR Operator nicht auflösen.
Ein Perzeptron mit linearer Aktivierungsfunktion ist jedoch nur ein linearer Klassifikator und kann dementsprechend nur linear separierbare Probleme, wie das Problem in Abbildung~\ref{subfig:linearSep}, lösen.
Linear nicht separierbare Probleme, wie zum Beispiel der XOR-Operator oder das in Abbildung~\ref{subfig:nonlinearSep} gezeigte Problem, können nicht korrekt abbildet werden.
Dies haben Marvin Minsky und Seymour Papert 1969 in ihrem einflussreichen Buch \textit{Perceptrons: an introduction to computational geometry} beschrieben~\cite[Unterabschnitt 1.2.1]{Goodfellow-et-al-2016}.
% \todo[inline]{(1) lineare Trennung mit perzeptron (2) xor-problem (3) nichtlineare trennung mit dem mehrschichtigen netz}
Solche nicht linear separierbare Probleme können jedoch gelöst werden, wenn mehrere Perzeptrons hintereinandergeschaltet werden.


\begin{figure}[h]
	\centering
	\begin{subfigure}[t]{0.4\textwidth}
		\includegraphics[width=\textwidth]{linearSep.pdf}
		\caption{Lineare Lösung für ein linear separierbares Problem.}
		\label{subfig:linearSep}
	\end{subfigure}
	\quad
	\begin{subfigure}[t]{0.4\textwidth}
		\includegraphics[width=\textwidth]{nonlinearSep.pdf}
		\caption{Nichtlineare Lösung für ein nicht linear separierbares Problem.}
		\label{subfig:nonlinearSep}
	\end{subfigure}
	\caption{Vergleich zwischen einem linear separierbaren und einem nicht linear separierbaren Problem. Angepasst von \cite{LinSep}.}
	\label{fig:linearNonlinearVgl}
\end{figure}

\subsection{Feedforward Netze}

% \textbf{absatz über Feedforward Netze. Basic}
% \color{blue}
% \begin{itemize}
% 	\item Definition (keine Kreise oder Schleifen). (Kommentar Florian: \"Rückkopplung\")
% 	\item Gegenstück zum RNN
% 	\item grundlegende Architektur in Layers 
% 	\item Aktivierungsfunktionen in Layers
% 	\item Outputlayer: Verschiedene Aktivierungsfunktionen:
% 	\item Linear für regression, z.B. Softmax für Wahrscheinlichkeitsverteilung Softmax
% \end{itemize}
% \color{black}

Um die Limitation auf lineare Klassifikation eines einzelnen Perzeptrons zu überwinden, können mehrere solche Perzeptrons hintereinandergeschaltet werden. 
Die einfachste Möglichkeit dies zu tun wird als Feedforward Netz bezeichnet.
Mit diesem Begriff beschreibt man ein neuronales Netz, zwischen dessen Knoten keine Kreise beziehungsweise Schleifen existieren.
Die Informationen wandern in der Verarbeitungsrichtung von den Eingabeneuronen zu den Ausgabeneuronen.
Für gewöhnlich sind die einzelnen Knoten in Schichten, sogenannten Layers, organisiert.
Auf ein Eingabe- oder auch Input Layer folgen eine Menge an sogenannten Hidden Layers.
Den Abschluss bildet das Ausgabe- oder auch Output Layer.
Die Neuronen eines einzelnen Hidden Layers sind meist mit uniform bezüglich ihrer Verbindungen zum vorherigen und zum nächsten Layer und verwenden identische Aktivierungsfunktionen, wie sie in Abschnitt~\ref{sec:activationfuncs} beschrieben sind.

Durch das Hintereinanderschalten von mehreren Layers von Neuronen mit nichtlinearen Aktivierungsfunktionen können auch Probleme, die nicht linear-separierbar sind, gelöst werden.
Tatsächlich sind solche neuronalen Netze bewiesenermaßen universelle Funktionsapproximatoren -- 
sie können mit endlich vielen Neuronen in den Hidden Layers beliebige kontinuierliche Funktionen auf kompakten Subsets von \(\mathbb{R}^n\) approximieren \cite[Unterabschnitt 6.4.1]{Goodfellow-et-al-2016}.
Das heißt zwar, dass es theoretisch für jede solche Funktion ein Netz gibt, das sie beschreiben kann, in der Praxis muss man dieses Netz aber finden und dann auch trainieren können.
% \todo[inline]{mit Quelle belegen? Universal approximation theorem }

Über die in der Arbeit verwendeten Netze hinaus gibt es viele weitere Unterkategorien von neuronalen Netzen, die in verschiedensten Bereichen Verwendung finden.
Convolutional Neural Networks zeichnen sich durch eine spezielle Struktur aus sogenannten Convolutional Layers aus, die als Feature Extractor fungieren.
Diese Art von Netz findet unter anderem in der Bildverarbeitung Anwendung.
Ein Netz, in dem Kreise beziehungsweise Schleifen existieren, bezeichnet man als \textit{Rekurrentes Neuronales Netz}.
Diese Netze werden vor allem dann eingesetzt, wenn die zeitliche Abfolge der Eingaben relevant ist, 
da sie im Gegensatz zu Feedforward Netzen einen internen Zustand halten.

\subsection{Aktivierungsfunktionen}
\label{sec:activationfuncs}
% \todo[inline]{Entscheiden, ob diese Section weg soll, oder auf nur ReLU reduziert werden sollte}
Es gibt verschiedene Aktivierungsfunktionen, die für den Einsatz in neuronalen Netzen in Frage kommen.
Sie sind notwendig, da ohne eine Nicht-Linearität das Netz in eine lineare Regression kollabiert, 
die dann wiederum nicht in der Lage ist nicht-lineare Funktionen darzustellen.
Sowohl die Aktivierungsfunktion als auch ihre Ableitung sollte schnell zu berechnen sein, 
da dies im Rahmen des Trainings mit dem Backpropagation Algorithmus häufig geschieht 
und sonst beträchtlicher Rechenaufwand entsteht.
% \todo[inline]{das referenziert dinge von Backprop, die der Leser hier vielleicht noch gar nicht weiß - Nach hinten?}

In der Vergangenheit wurden einige verschiedene Aktivierungsfunktionen verwendet.
% Einige häufig verwendete Aktivierungsfunktionen sollen hier vorgestellt werden.
Jede dieser Funktionen stellt eine Nicht-Linearität dar und nimmt eine einzelne Zahl, wendet eine bestimmte, festgelegte mathematische 
Operation auf diese an und gibt das Ergebnis zurück.
Historisch ist häufig die Sigmoid-Funktion \(\sigma(x)\) verwendet worden, die folgendermaßen definiert ist:
\begin{equation}
	\sigma(x) = \frac{1}{1 + e^x} = \frac{e^x}{e^{x + 1}} \, .
	\label{func:Sigmoid}
\end{equation}
In der Praxis jedoch haben sich einige Nachteile der Sigmoid-Funktion und artverwandter Funktionen  gezeigt.
% \todo{Probleme mit Sigmoid genauer beschreiben? saturation}
Der schwerwiegendste davon ist die sogenannte Saturierung,
dass die Ableitung der Sigmoid-Funktion für betragsmäßig große Eingaben beinahe \(0\) ist.
Das sorgt für einen langsamen Lernprozess.
% Einige dieser Probleme konnten mit der Verwendung des Tangens hyperbolicus (tanh) behoben werden, 
% aber das Problem der Saturierung, dass ihre Ableitung bei großen Beträgen beinah \(0\) wird, nicht.
In letzter Zeit haben sich sogenannte \textit{Rectified Linear Units}, oder kurz ReLUs, durchgesetzt.
Diese sind folgendermaßen definiert:
% 
% \begin{description}
	
	% hier Könnte man die Absätze zu Sigmoid und TanH einfügen
	% \item[Sigmoid-Funktion] \hfill \\
		\begin{equation}
			f(x) = \frac{1}{1 + e^x} = \frac{e^x}{e^{x + 1}}
			\label{func:Sigmoid}
		\end{equation}
		\begin{equation}
			f'(x) = f(x) * (1 - f(x))
		\end{equation}
		Die mathematische Form der Sigmoid Aktivierungsfunktion ist in Abbildung \ref{sigmoidFunc} zu sehen.
		Sie bildet die reellen Zahlen \(\mathbb{R}\) auf das Intervall \((0,1)\) ab. 
		Für betragsmäßig größer werdende negative Zahlen nähert sich der Rückgabewert \(0\) an,
		ebenso wie für größer werdende positive Zahlen sich der Rückgabewert an \(1\) annähert.

		Die Sigmoid Funktion ist eine historisch häufig genutze Funktion, da sie das Verhalten eines natürlichen Neurons,
		der biologischen Motivation für künstliche Neuronen, gut nachbildet:
		komplette Inaktivität eines Neurons bei Ausgabe 0 bis zum feuern mit maximaler Frequenz bei Ausgabe 1.

		In der Praxis jedoch haben sich einige Nachteile der Sigmoid Funktion gezeigt, weshalb sie quasi nicht mehr genutzt wird.
		Der gewichtigste von diesen ist, dass ihre Ableitung bei großen Beträgen beinah \(0\) ist.
		Dies führt dazu, dass während der Ausführung des Backpropagation-Algorithmus beinah keine Änderungen passieren und dementsprechend das Netz sehr langsam lernt.
		
		\begin{figure}
			\centering
			\includegraphics[width=0.618\textwidth]{Sigmoid}
			\caption{Plot der Sigmoid Funktion}
			\label{sigmoidFunc}
		\end{figure}
		  


	\item[TanH] \hfill \\
		\begin{equation}
			f(x) = \tanh(x) = \frac{e^x - e^{-x}}{e^x + e^{-x}}
		\end{equation}
		\begin{equation}
			f'(x) = 1 - f(x)^2
		\end{equation}
		Die TanH-Aktivierungsfunktion ist in Abbildung \ref{tanhfunction} dargestellt.
		Im Gegensatz zur Sigmoid Funktion bildet sie die reellen Zahlen \(\mathbb{R}\) auf das Intervall \((-1, 1)\) ab.
		Weil sie zentriert um den Nullpunkt ist, wird sie bei realen Anwendungen der Sigmoid Funktion vorgezogen.
		Das Saturationsproblem der Sigmoid Funktion besteht jedoch immer noch.
		\begin{figure}
			\centering
			\includegraphics[width=0.618\textwidth]{Tanh}
			\caption{Plot der Tanh Funktion}
			\label{tanhfunction}
		\end{figure}
	
	% \item[ReLU] \hfill \\
\begin{equation}
	f(x) = \max(0, x) \, ,
\end{equation}
\begin{equation*}
	f'(x) = \begin{cases}
	0 &\text{, falls $x < 0$}\\
	1 &\text{, falls $x > 0$}
	\end{cases} \, .
\end{equation*}
% \todo[inline]{sicherstellen dass das nicht so über die seite umgebrochen wird}

\begin{figure}[h]
	\centering
	\includegraphics[width=0.618\textwidth]{reluPlot.pdf}
	\caption{Schaubild der Ausgabe einer ReLU.}
	\label{reluoutput}
\end{figure}
Abbildung~\ref{reluoutput} zeigt das Schaubild einer solchen ReLU. 
% \todo{ReLU plot neu machen: dicker bei unter 0 und als Vector graphic statt als PNG}
Die Aktivierung von ReLUs ist ein einfacher Schwellwert, der weit weniger rechenintensiv ist als die aufwendigen Exponentialfunktionen von Sigmoid und tanh.
In der Praxis hat sich zudem gezeigt, dass ReLUs oft deutlich schneller konvergieren als Sigmoid- oder tanh-Neuronen, da sie kein Problem mit Saturierung haben.
Krizhevsky et al. haben in ihrem Paper~\cite{NIPS2012_4824} einen Geschwindigkeitsgewinn um Faktor 6 feststellen können.

Ein Problem, das mit ReLUs jedoch existiert ist, dass einzelne Neuronen während dem Training "absterben" können, falls sie irgendwann in dem Bereich landen, in dem der Gradient 0 ist.
Diese Neuronen sind dann für jeden beliebigen Input inaktiv und können niemals wieder etwas zur Ausgabe des Netzes beitragen.
Durch die Wahl einer geeigneten Lernrate oder den Einsatz sogenannter Leaky ReLUs lässt sich dies jedoch vermeiden.
Leaky ReLUs haben im Gegensatz zu normalen ReLUs eine kleine positive Steigung im negativen Bereich.
\begin{equation}
	f(x) = \begin{cases}
		x &\text{, falls } x  >  0\\
		0.01 x &\text{, falls } x  \leq  0
	\end{cases}
\end{equation} 

Je nachdem, welche Aufgabe das neuronale Netz erfüllen soll, können unterschiedliche Aktivierungsfunktionen für das Output Layer Sinn ergeben.
Häufig genutzt ist die \textit{Softmax}-Funktion für Klassifikation, da sie eine Wahrscheinlichkeitsfunktion als Ausgabe hat.
Für Regression benutzt man lineare Aktivierungsfunktionen für das Output Layer. 




% \end{description}

% \begin{figure}[h]
%     \centering
%     \begin{subfigure}[t]{0.3\textwidth}
% 		\includegraphics[width=\textwidth]{Sigmoid}
% 		\caption{Sigmoid Funktion}
%     \end{subfigure}
%     \begin{subfigure}[t]{0.3\textwidth}
% 		\includegraphics[width=\textwidth]{Tanh}
% 		\caption{TanH Funktion}
%     \end{subfigure}
%     \begin{subfigure}[t]{0.3\textwidth}
%         \includegraphics[width=\textwidth]{ReLu}
%         \caption{ReLU}
%     \end{subfigure}
%     \caption{Häufig verwendete Aktivierungsfunktionen}
%     \label{eval:function}
% \end{figure}





% \subsection{Backpropagation}

% \todo[inline]{Optional: am Ende entscheiden ob ich noch mehr Kontent möchte und dann gegebenenfalls ausformulieren}
% Der Backpropagation Algorithmus ist ein Verfahren, mit denen künstliche neuronale Netze in der Lage sind, komplizierte Zielfunktionen einzulernen.
% Es ist eine Methode, bei der effizient der Gradient der Fehlerfunktion in Abhängigkeit vom Gewicht der einzelnen Kanten im Netz bestimmt werden kann,
% was dann für einen Gradientenabstieg verwendet werden kann. 

% \color{blue}
% \begin{itemize}
% 	\item Definition und Beschreibung
% 	\item Nur supervised learning: Gradient der Fehlerfunktion wird benötigt \(\rightarrow\) Tatsächliches Ergebnis muss bekannt sein.
% 	\item "Finden einer Funktion, die am besten die Inputs auf die outputs mapt"
% \end{itemize}
% \color{black}

\subsection{Performance-Maß}

% Neuronale Netze werden in der Regel danach bewertet, wie gut sie mit unbekannten, das heißt nicht im Training vorgekommenen, Daten umgehen können.
Neuronale Netze werden in der Regel danach bewertet, wie gut sie mit unbekannten Daten umgehen können.
Trainingsdaten bezeichnen dabei die Daten, die während dem Trainings dazu benutzt werden, die einzelnen Gewichte des Netzes anzupassen.
Testdaten sind ein von den Trainingsdaten disjunktes Datenset, auf denen die Qualität der Ausgaben des Netzes überprüft wird.
% Das essenzielle Maß nachdenen neuronale Netze bewertet werden, ist wie gut sie mit neuen, unbekannten Daten umgehen, die nicht in den Trainingsdaten vorhanden waren. 
Diese Eigenschaft von Trainingsdaten auf unabhängige Testdaten schließen zu können wird Generalisierung genannt. 

Als Overfitting bezeichnet man es, wenn ein gelerntes Modell sich zu sehr an den zum gegebenen Training betrachteten Datensatz anpasst und 
dafür in Kauf nimmt, zusätzliche oder zukünftige Daten schlechter zu repräsentieren.
Das Modell wird also schlechter darin zu generalisieren.
Speziell im Feld der neuronalen Netze ist Overfitting meist daran zu erkennen,
dass sich die Qualität der Ausgaben des Netzes auf dem Trainingsdatenset weiter verbessert,
während sie auf dem Testdatenset schlechter wird.
Dies kann zum Beispiel der Fall sein, wenn das Modell Rauschen in den Trainingsdaten als Teil der zugrundeliegenden Struktur interpretiert. 


Dem Overfitting gegenüber steht das Underfitting. 
Als Underfitting bezeichnet man wenn das Netz nicht in der Lage ist
eine ausreichend gute Performance auf den Trainingsdaten zu erreichen.
Das kann passieren, wenn das Modell nicht ausreichend komplex ist, um die zugrundeliegende Struktur der Daten abzubilden.
Sowohl Overfitting als auch Underfitting hängen mit der Kapazität eines Netzes zusammen.
Ist die Kapazität zu gering, kann es sein, dass das Netz daran scheitert die Trainingsdaten zu lernen.
Ist die Kapazität zu groß, so kann es passieren, dass das Netz gewissermaßen die Trainingsdaten einfach auswendig lernt.
% \todo{umgangsprachlich mit einer besseren Formulierung ersetzen...}
Dies ist beispielhaft in Abbildung~\ref{fig:capacity} zu sehen.
Aus der zugrundeliegenden Cosinus Funktion werden Stichproben mit einem Rauschterm entnommen. 
Stellvertretend für das Lernen mit neuronalen Netzen wird hier lineare Regression benutzt,
um die Parameter eines Polynoms zu lernen.
Die Kapazität des Modells wird hier durch den Grad des Polynoms festgelegt.

In Abbildung~\ref{subfig:underfitting} wird ein Polynom mit dem Grad 1 gelernt.
Die Kapazität ist zu niedrig und dementsprechend schafft das Modell es nicht die Stichproben akkurat zu repräsentieren.
In Abbildung~\ref{subfig:rightfitting} ist der Grad des Polynoms 4.
Das resultierende Modell ist eine gute Approximation der ursprünglichen Funktion.
In Abbildung~\ref{subfig:overfitting} ist der Grad des Polynoms mit 17 deutlich zu hoch.
Man kann am resultierenden Modell erkennen, wie es dem Rauschen der Samples folgt und wie es an den Rändern des betrachteten Bereichs schnell divergiert.

\begin{figure}[p]
    \centering
	
	\begin{subfigure}[t]{0.6\textwidth}
		\includegraphics[width=\textwidth]{plotUnderfitting.pdf}
		\caption{Underfitting mit Polynom von Grad 1}
		\label{subfig:underfitting}
	\end{subfigure}
	% \quad
	\begin{subfigure}[t]{0.6\textwidth}
		\includegraphics[width=\textwidth]{plotPassendesModell.pdf}
		\caption{Passendes Modell -- ein Polynom von Grad 4}
		\label{subfig:rightfitting}
	\end{subfigure}
	% \quad
	\begin{subfigure}[t]{0.6\textwidth}
        \includegraphics[width=\textwidth]{plotOverfitting.pdf}
		\caption{Overfitting mit Polynom von Grad 17}
		\label{subfig:overfitting}
	\end{subfigure}
	\caption{Visualisierung von Modellen mit unterschiedlichen Kapazitäten.}
	\label{fig:capacity}
\end{figure}

Die beste Methode, um Overfitting zu vermeiden, ist es, mehr Trainingsdaten zu benutzen.
In der Praxis kann dies oft umständlich, kostspielig oder sogar unmöglich sein, 
weshalb auf das Generieren von synthetischen Trainingsdaten, auch Datenaugmentierung genannt, zurückgegriffen wird.
% There is no Data like more Data
Auch der Einsatz von Regularisierungsverfahren kann helfen Overfitting zu verringern.
Als letzter Schritt kann die Kapazität der Architektur gesenkt werden,
zum Beispiel, indem die Anzahl der zur Verfügung stehenden Layer reduziert wird.


% \color{blue}
% \begin{itemize}
% 	\item Methoden um Overfitting zu vermeiden:
% 	\item Mehr Trainingsdaten (z.\,B. durch Data-Augmentation)
% 	\item Regularisierung (L1, \(L_2\) (siehe unten), dropout)
% \end{itemize}
% \color{black}

\subsection{Regularisierung} \label{ssec:Regul}

% \color{blue}
% \begin{itemize}
% 	\item Regularisierung Definition
% 	\item Mathematische Formel Darstellung
% 	\item \(L_1\) und \(L_2\) Regularisierung
% 	\item L0 - warum nicht?
% 	\item (Maybe Dropout)
% 	\item Early Stopping? (das internet sagt, es ist eine art von Regularisierung und falls ich sie verwenden sollte, sollte ich sie hier erwähnen)
% \end{itemize}
% \color{black}

Als Regularisierung bezeichnet man Techniken, die zur Vermeidung von Overfitting verwendet werden.
% Sie wird in der Hoffnung angewendet, dass das Modell mit Regularisierung besser generalisiert als ohne.
Viele Techniken des maschinellen Lernens funktionieren so, dass sie den Fehler auf den Trainingsdaten reduzieren und damit potenziell indirekt auch den auf den Testdaten.
Im Gegensatz dazu soll hier primär der Fehler auf den Testdaten reduziert werden, selbst wenn das zu schlechteren Ergebnissen auf den Trainingsdaten führt.
Das Ziel ist also bessere Generalisierung.

Eine Möglichkeit ist, dass zur Loss-Funktion ein Regularisierungsterm \(R\) hinzugefügt wird, 
der die Kosten basierend auf der Komplexität des Modells erhöht.
%  
\begin{equation}
	\min_f \sum\limits_{i=1}^{m} V(f(\vec{x}_i), \vec{y}_i) + \lambda R(f)
\end{equation} 
 
Dabei ist \(V\) die Loss-Funktion, beispielsweise \textit{Mean-Square-Error} oder \textit{Mean-Absolute-Error}.
\(m\) ist die Anzahl der Feature-Label-Paare,
\(x_i\) ist ein Vektor, der die Eingabefeatures enthält und \(y_i\) ist das dazugehörige Label.
Die Funktion \(f\) ist in unserem Fall das neuronale Netz, das die Features entgegennimmt.
\(\lambda\) ist ein Parameter, der die Gewichtung des Regularisierungsterm festlegt.
Wählt man diesen Parameter zu klein, so kann es sein, dass das Modell trotz Regularisierung noch immer overfittet.
Wählt man ihn zu groß, so kann es sein, dass das Modell das Problem nicht mehr korrekt abbildet und es zu Underfitting kommt.

Der Regularisierungsterm \(R\) wird so gewählt, dass er die Komplexität der Funktion \(f\) widerspiegelt.
Ein gutes Maß für die Komplexität eines neuronalen Netzes sind die Gewichte zwischen den Neuronen.
Beispiele für \(R\) wären zum Beispiel die \(L_1\)- oder die \(L_2\)-Regularisierung. % die jeweils mit der \(L_1\) beziehungsweise mit der \(L_2\)-Norm arbeiten.
Der entscheidende Unterschied zwischen den beiden ist der unterschiedlicher Strafterm, zu sehen in Gleichung~\eqref{eq:MSE-L1} für \(L_1\) und Gleichung~\eqref{eq:MSE-L2} für \(L_2\). 
Die Fehlerfunktion ist jeweils MSE mit dazugehörigen Strafterm. \(\mat{W}_{j,k}\) ist eine Matrix der Gewichte zwischen den einzelnen Neuronen. 
% 
\begin{align}
	J(X, Y) &= \frac{1}{m} \sum_{i=1}^{m} (\vec{y}^{(i)} - \hat{\vec{y}}^{(i)})^2 + \sum_{j, k} (|\mat{W}_{j,k}|) \label{eq:MSE-L1} \\
	J(X, Y) &= \frac{1}{m} \sum_{i=1}^{m} (\vec{y}^{(i)} - \hat{\vec{y}}^{(i)})^2 + \sum_{j, k} (\mat{W}_{j,k}^2) \label{eq:MSE-L2}
\end{align}
% 
% \begin{equation} \label{eq:MSE-L1}
% 	J(X, Y) = \frac{1}{m} \sum_{i=1}^{m} (\vec{y}^{(i)} - \hat{\vec{y}}^{(i)})^2 + \sum_{j, k} (|\mat{W}_{j,k}|)
% \end{equation} 

% \begin{equation} \label{eq:MSE-L2}
% 	J(X, Y) = \frac{1}{m} \sum_{i=1}^{m} (\vec{y}^{(i)} - \hat{\vec{y}}^{(i)})^2 + \sum_{j, k} (\mat{W}_{j,k}^2)
% \end{equation} 


Ein Regressionsmodell, das \(L_1\)-Regularisierung verwendet, wird auch als Lasso Regression bezeichnet, 
während ein Modell mit \(L_2\)-Regularisierung als Ridge Regression beschrieben werden kann.
Vergleicht man die beiden Ansätze, so schrumpft die \(L_1\)-Norm weniger wichtige Gewichte auf 0, was zu dünn besetzten Gewichtsvektoren führt.
Dies kann eine wünschenswerte Eigenschaft sein.
Im Gegensatz dazu hat die \(L_2\)-Regularisierung den Vorteil, dass sie effizienter berechnen werden kann.
Der Strafterm von \(L_2\) hat eine geschlossene Form und kann in Form einer Matrix angewendet werden, während die Funktion von \(L_1\) auf Grund des Betrags eine nicht-differenzierbare ist.
Außerdem wird die Existenz von großen Gewichten stärker bestraft als bei der \(L_1\)-Regularisierung.

% \todo{absatz zu (L0 und warum man es nicht benutzt?)}

Eine weitere Regularisierungstechnik ist Dropout~\cite{JMLR:v15:srivastava14a}.
Dabei werden während dem Training eines neuronalen Netzes 
mit einer festgelegten Wahrscheinlichkeit zufällig Neuronen und die dazugehörigen Verbindungen abgeschaltet, 
wie in Abbildung~\ref{fig:dropout} dargestellt.
% \todo{Ist die Grafik essenziell fürs Verständnis?}
Dies soll insofern Overfitting vermeiden, dass es übermäßige Coadaption von mehreren Neuronen erschwert.
Dropout als Technik wird insbesondere bei tiefen neuronalen Netzen mit einer hohen Anzahl von Hidden Layers eingesetzt. 
Als Early Stopping wird eine Technik bezeichnet, bei der die Regularisierung durch das frühzeitige Beenden des Trainings erreicht wird \cite[Kapitel 7.8]{Goodfellow-et-al-2016}.
Dabei wird versucht abzuschätzen, an welchem Punkt des Trainings 
eine Verbesserung der Ergebnisse auf dem Trainingsset zu Lasten der Ergebnisse auf dem Testset stattfindet.


\begin{figure}[h]
    \centering
    \begin{subfigure}[t]{0.4\textwidth}
		\includegraphics[width=\textwidth]{neuralNet}
		\caption{Unverändertes neuronales Netz}
    \end{subfigure}
    \begin{subfigure}[t]{0.4\textwidth}
		\includegraphics[width=\textwidth]{neuralNet_dropped}
		\caption{Bei der Anwendung von Dropout.}
	\end{subfigure}
    \caption{Beispiel für die Dropout-Regularisierung~\cite{JMLR:v15:srivastava14a}.}
    \label{fig:dropout}
\end{figure}





\section{TableSort System}

% \todo[inline]{Viel stuff über das TableSort System}
% \color{blue}

% \begin{itemize}
% 	\item kleiner, experimenteller Bandsortierer~\cite{doll2015}
% 	\item Entstanden in Kooperation zwischen dem Fraunhofer IOSB, Abteilung Sichtprüfsysteme, und dem Institut für Intelligente Sensor Aktor Systeme des Karlsruher Institut für Technologie.
% 	\item Im Rahmen des \textit{TrackSort} Projekts
% 	\item Gedacht für Experimente, wenn es zu aufwendig ist das mit dem großen großen zu machen und zum Mitnehmen auf Messen.
% 	\item 2 Modi: mit Förderband und mit Rutsche
% 	\item Mit Flächenkamera für TrackSort als auch die Zeilenkamera sind dargestellt.
% 	\item Ringlicht (Refence später)
% 	\item Die Zeilenkamera wird zurzeit in industriellen Schüttgutsortieranlagen verwendet, ist aber nicht optimal (Siehe all die Literatur)
% \end{itemize}
% \color{black}


Der \textit{TableSort}-Schüttgutsortierer ist eine modulare Versuchsplattform, die konzipiert wurde, um neue Schüttgutsortierkonzepte in einem kleineren Rahmen experimentell erproben zu können.
Industrielle Schüttgutsortieranlagen sind sehr groß und etwas an ihrer Konfiguration zu ändern ist sowohl zeitaufwendig als auch arbeitsintensiv.
Daher ist es sinnvoll, eine Plattform zu haben, die flexibel und transportabel ist und leicht umgebaut werden kann.
Das System ist im Rahmen des \textit{TrackSort}-Projekts in Kooperation zwischen dem Fraunhofer IOSB, Abteilung Sichtprüfsysteme, und dem Lehrstuhl für Intelligente Sensor Aktor Systeme des Karlsruher Institut für Technologie entstanden~\cite{doll2015}.
Es kann in zwei verschiedenen Transportkonfigurationen benutzt werden.
Einmal werden die Schüttgutpartikel mittels eines Förderbands beschleunigt.
Diese Konfiguration ist in Abbildung~\ref{fig:tablesortsystem} zu sehen.
Dazu kommt die Möglichkeit, das Förderband durch eine Rutsche zu ersetzen, sodass die Schüttgutpartikel stattdessen durch die Schwerkraft beschleunigt werden. 

Im Rahmen dieser Arbeit wurde die Flächenkamera-Konfiguration mit einem Ringlicht zur Beleuchtung verwendet.
Schematisch ist der Ablauf der Schüttgutsortierung in Abbildung~\ref{fig:aufbau_tablesort} zu sehen.
Die Funktionsweise des Systems wird im Folgenden erklärt.
Das zu sortierende Schüttgut wird durch einen Schwingförderer nach und nach über eine Rutsche auf das Förderband aufgebracht.
Dort wird es in Richtung des Druckluftdüsenarrays beschleunigt. 
Auf das vom Ringlicht beleuchtete Ende des Förderbands ist die Flächenkamera gerichtet.
Kurz nachdem die Schüttgutpartikel die Flugphase beginnen, passieren sie das Druckluftdüsenarray.
Dort werden sie, basierend auf ihrer Klassifizierung durch die Verarbeitung der Bilddaten von der Kamera, aussortiert oder passieren gelassen.

\begin{figure}[h]
	% \missingfigure{Bild von TablesortSystem}
	\includegraphics[width=\textwidth]{TrackSortPic}
	\caption{TableSort Schüttgutsortiersystem \cite{fraunhoferiosb2017}.}
	% \todo{Quelle Bild!}
	\label{fig:tablesortsystem}
\end{figure}


\begin{figure}[h]
    \centering
    \def\svgwidth{\columnwidth}
	\input{img/Aufbau-moved.pdf_tex}
	\caption[Schematische Darstellung des \textit{TableSort}-Systems nach~\cite{Pfaff2017}.]{
		Schematische Darstellung des optischen Bandsortierers \textit{TableSort} nach~\cite{Pfaff2017}.
	}
	\label{fig:aufbau_tablesort}

\end{figure}



    \section{Stand der Technik}
\label{cap:relWork}

% Notizen:
% \todo[inline]{Das aller aller dickste TODO}
% \color{blue}
% Wie wird das, was ich dann mit Neuronalen Netzen machen möchte aktuell gemacht?


% Teilaspekt vom Tracksort Multi-Targettracking: Bewegungsprädiktion.
% Primär Florian's Dissertation: Kapitel 4.


% \begin{itemize}
%     \item Im Rahmen dieser Arbeit es geht um die Prädiktion der Bewegung von Schüttgutpartikeln.
%     \item "Bewegungsmodell" bei Zeilenkamera: Identical Delay, weil nur eine Beobachtung.
%     \item Modelle erst seit Flächenkamera notwendig.
%     \item Vorraussetzung: Trackassoziationen.
%     \item Aber: wir nehmen dieses Problem als gelöst an.
%     \item In der Realität ist es das nicht (MA Tobi K), reale daten sind nicht zwingend 100\% genau, aber wir arbeiten mal damit.
%     \item in Florians Diss: mehrere Modelle beschrieben zum prädizieren:
%     \item CV und CA, und was die dahinterstehenden Annahmen sind.
%     \item CV: Konstante Geschwindigkeit - Perfekte Beruhigung
%     \item CA: Konstante Beschleunigung - Geschwindigkeit ändert sich Konstante
%     \item Dazu kommen noch Szenario spezifische Modelle, bei denen gezeigt wurden, dass sie für das Szenario besser sind.
%     \item CVBC, IA - upgrades.
%     \item Vorgriff ins evaluationskapitel? - dort wird dann behandelt wie man sie Mathematisch beschreibt. 
% \end{itemize}

% \color{black}


Im Rahmen dieser Arbeit geht es um die Prädiktion der Bewegung von Schüttgutpartikeln.
Der Einsatz von verschiedenen Bewegungsmodellen ist erst für Schüttgutsortierer mit Flächenkamera sinnvoll.
Eine Zeilenkamera liefert nur einen einzelnen Datenpunkt bezüglich Zeit und Position eines Partikels.
In~\cite{Pfaff2018} wird unter anderem ein Bewegungsmodell beschrieben, das das Verhalten eines Schüttgutsortierers mit Zeilenkamera emuliert.
Wie in schon in Abbildung~\ref{fig:predMissed} dargestellt wurde, muss angenommen werden, dass es zu keinerlei Bewegung orthogonal zur Transportrichtung kommt.
Für die Prädiktion des Zeitpunkts wird die durchschnittliche Zeit bestimmt, die ein Partikel von der Position der Zeilenkamera zur Position des Druckluftdüsenarrays benötigt 
und diese als konstanter Offset für jede Partikeldetektion angenommen.

Um den Separationsprozess durch den Einsatz von prädiktiven Tracking Methoden und Bewegungsmodellen zu verbessern ist eine Assoziation 
der beobachteten Partikelpositionen zu tatsächlichen Tracks notwendig. 
Im Rahmen dieser Arbeit wird dieses Problem nicht betrachtet. 
Es wird direkt mit den assoziierten Trackdaten gearbeitet, 
obwohl dieser Assoziationsprozess noch Gegenstand aktueller Forschung ist.
\todo{hier schon erwähnen, dass die Assoziation auf den selbst gesammelten Daten vielleicht flawed ist?}

Die grundlegenden Bewegungsmodelle, die in \cite{Pfaff2018} beschrieben werden sind 
einerseits das Constant Velocity Modell und andererseits das Constant Acceleration Modell.
Das Constant Velocity Modell prädiziert die Bewegung eines Teilchens unter der Annahme, dass es sich mit einer konstanten Geschwindigkeit bewegt.
Basierend auf den letzten zwei bekannten Positionen des Partikels wird dessen Geschwindigkeit entlang der beiden Achsen bestimmt 
und davon die zukünftige Bewegung abgeleitet.
Diese Annahme ist jedoch nicht immer korrekt.
Es kann sein, dass bei einem Bandsortierer das Förderband nicht lang genug ist, um das Schüttgut komplett zu beruhigen.
Dann haben die Teilchen eine Beschleunigung, die nicht 0 ist.
Das Constant Acceleration Modell dahingegen prädiziert die Bewegung des Teilchens unter der Annahme, dass es sich mit einer konstanten Beschleunigung bewegt.
Diese Beschleunigung wird anhand der letzten drei bekannten Positionen bestimmt.
Anhang dieser Beschleunigung und der aktuellen Geschwindigkeit werden die zukünftigen Positionen abgeleitet.

In \cite{Pfaff2018} werden weitere, szenariospezifische Bewegungsmodelle beschrieben.
Bei dem sogenannten Bias-Corrected Constant Velocity Modell wird das Constant Velocity Modell als Grundlage genommen und ein Korrekturterm eingeführt.
Basierend auf den zuvor beobachteten Schüttgutpartikeln wird ein durchschnittlicher temporaler Bias bestimmt, der von den zukünftigen Prädiktionen abgezogen wird.
Die Annahme ist hier, dass der Bias von zukünftige Partikel ähnlich zu dem der zuvor beobachteten Partikeln sein wird.
Beim Identical Acceleration Modell wird ebenfalls ein Korrekturterm benutzt, 
der im Gegensatz zum Bias-Corrected Constant Velocity Modell jedoch nicht absolut, sondern abhängig von der letzten bekannten Position des Partikels ist.
Auf jedem der zuvor beobachteten Partikeln wird der Wert einer zusätzlichen Beschleunigung bestimmt, die zu einer optimalen Prädiktion führen würde und dann der Durchschnitt dieser Beschleunigungen gebildet.
Dieser wird dann als Korrekturterm auf ein Constant Velocity Modell addiert, sodass sich eine Formel ähnlich zu der des Constant Acceleration Modelles ergibt.
Um das Verhalten von den Partikeln, deren Geschwindigkeit sich der des Förderbands nähert ohne sie zu überschreiten, besser abzubilden als durch ein Constant Acceleration Modell,
wird das Constant Acceleration with Limited Velocity Modell beschrieben. 
Dabei wird die Geschwindigkeit des Förderbands bestimmt und Partikel, die diese erreichen, von dann aus mit konstanter Geschwindigkeit weiter prädiziert.
Es wurden zudem zwei Modelle, Constant Acceleration Disallowing Sign Change Modell und Ratio-Based Deceleration Modell, vorgestellt, 
die spezifisch konzipiert wurden um die Bewegungprädiktion orthogonal zur Transportrichtung zu verbessern.


    \chapter{Daten}

Wie bei jeder Anwendung von maschinellen Lernverfahren sind die zugrundeliegenden Daten von äußerster Wichtigkeit.
Im Rahmen dieser Masterarbeit wurden zu Beginn existierende Aufnahmen und Track benutzt, 
bevor dann selbst am Schüttgutsortierer des Fraunhofer IOSBs Aufnahmen gemacht wurden.


\section{Eigene Aufnahmen}

\subsection{Kamera}

[Beschreibung von der Bonito Kamera, stats usw.]

\subsection{Schüttgüter}

Aufgenommen wurden vier verschiedene Schüttgüter:

\begin{itemize}
    \item Kugeln
    \item grüne Pfefferkörner
    \item Zylinder
    \item Weizenkörner
\end{itemize}

Die Kugeln und der Pfeffer sowie die Zylinder und die Weizenkörner bilden jeweils 
ein Paar aus einem geometrischen Körper und einem echten Objekt, das grob dessen Form ähnelt.
[Details]

\subsection{Pipeline}

Die Bilder wurden in Batches von je 3500 gesammelt.
Die Bonito Kamera schreibt sie als eine Bayer-Matrix in Bitmap Dateien.
Auf Grund der Menge an Bildern war es sinnvoll die Dateien in das png Format zu übertragen.
Die Features, die für das Trainieren der Netze benutzt werden, sind die Koordinaten der Mittelpunkte der Objekte.
Um diese zu bestimmten müssen zunächst die Dateien mittels \textit{Demosaicing} rekonstruiert werden um gewöhnliche RGB Bilder zu erhalten.
[Reference Debayer script?]
Auf diesen kann dann eine Segmentierung vorgenommen werden.
Hierzu wurde die Computer Vision Library OpenCV benutzt.
Das Ergebnis des Segmentierungsscripts [Reference segement.py] ist ein CSV File für jedes Batch.
Eine Zeile repräsentiert ein Bild aus dem Batch, also einen Zeitschritt.
Zu Beginn jeder Zeile steht zunächst die Frame Nummer, gefolgt von der Anzahl der detektierten Partikel. 
Die X- und Y-Koordinaten von der detektierten Partikeln sind dann links angehängt

\begin{table}[ht]
\begin{adjustbox}{width=1\textwidth}
\begin{tabular}{c|c|c|c|c|c}%
    %\bfseries Person & \bfseries Matr.~No.% specify table head
    
    \bfseries TrackID\_1\_X & \bfseries TrackID\_1\_Y & \bfseries TrackID\_2\_X  & \bfseries TrackID\_2\_Y & \bfseries TrackID\_3\_X & \bfseries TrackID\_3\_Y
    \csvreader[head to column names]{docExample.csv}{}% use head of csv as column names
    {\\\hline\csvcoli&\csvcolii&\csvcoliii&\csvcoliv&\csvcolv&\csvcolvi} % specify your coloumns here
    \end{tabular}
\end{adjustbox}
\caption{Beispielhafter Ausschnitt aus einem CSV File} 
\end{table} 

\subsection{Menge}

Insgesamt wurden 177954 Bilder aufgenommen.

Es wurden 
7002 Kugeln in 14 Batches,
7056 Pfefferkörner in 13 Batches,
17049 Zylinder in 11 Batches
und 8549 Weizenkörner in 13 Batches aufgenommen.

bei einer FeatureSize von 5 ergeben sich bei den Kugeln so 98.966 Feature-Label Paare,
bei den Pfefferkörnern 105.101 Feature-Label Paare,
bei den Zylindern 244.422 Feature-Label Paare
und bei den Weizenkörner 132.140 Feature-Label Paare.


    \chapter{Umsetzung und Implementierung}

\section{Software}

requirements.txt kann im Anhang gefunden werden mit der vollständigen Liste.

\begin{itemize}
    \item Virtual Python environment.
    \item Implementiert in Tensorflow. (angefangen in version 1.8, später nach 1.11 upgedatet)
    \item Datenhandling: mit Pandas. Da Data Science ein wichtiger Part der Arbeit war, sehr wichtig erwähnen
    \item Matplotlib für Visualisierung (die meisten selbstgemachten grafiken hier in der Arbeit)
    \item OpenCV für Bilderdinge in der Pipeline (wie oben erwähnt)
    \item MATLAB, für Tracksort und die Ursprünglichen implementation der Vergleichsdinge für evaluationen 
\end{itemize}

\todo[inline]{Aufpassen dass das nicht zu viel wird}

\section{Code Struktur}


\section{Hyperparameter}

Hyperparameter sind die Variablen, die die Struktur des Netzes bestimmen (Eg: Anzahl Layers, FeatureSize) 
sowie die Variablen, die festlegen wie das Netz trainiert (z.B. Lernrate, Anzahl Epochen)
 determine how the network is trained(Eg: Learning Rate).

Hyperparameters are set before training(before optimizing the weights and bias).

\noindent\begin{minipage}{\textwidth}
\begin{lstlisting}[language=json,firstnumber=1, caption={Beispiel eines Hyperparameter Files in JSON}, captionpos=b, label=lst-hyperparam]
    {
        "arch": {
            "dropout_rate": 0.0,
            "hidden_layers": [16, 16, 16],
            "feature_size": 5,
            "activation": "leaky_relu"
        },
        "problem": {
            "data_path": "/home/hornberger/MasterarbeitTobias/data/simulated/SpheresDownsampled",
            "modelBasePath": "/home/hornberger/MasterarbeitTobias/models/simulated/",
            "imagePath": "/home/hornberger/MasterarbeitTobias/images/",
            "separator": 0,
            "separatorPosition": 1550,
            "thresholdPoint": 1200,
            "predictionCutOff": 1300

        },
        "train": {
            "batch_size": 1000,
            "epochs": 500,
            "steps_per_epoch": 200,
            "learning_rate": 0.01,
            "optimizer": "Adam"
        },
        "data": {
            "numberFakeLines": 500,
            "testSize": 0.1,
            "augmentMidpoint": 1123,
            "augmentRange": 1000,
            "direction": "x",
            "unitLoc": "px",
            "unitTime": "1/100 Frames",    
            "limits": [0.388, 0.788, 0.0, 0.18]
        }
    }

    
\end{lstlisting}
\end{minipage}
    
\begin{itemize}
\item Architektur:
    \begin{itemize}
        \item Dropout: Wahrscheinlichkeit für das zufälliges ausschalten von einzelnen Neuronen
        \item Hidden Layer: ein Array an Zahlen repräsentiert die Architektur der Hidden Layers. Jede Zahl ist ein FC Layer mit so vielen neuronen
        \item FeatureSize: Wie viele Positionen bekommt das Netz als Input ( => Größe des Inputlayers = 2x FeatureSize)
        \item Activation: Aktivierungsfunktionen für die neuronen der Hidden Layers
        % - optionen: "relu", "leaky_relu", "linear"
    \end{itemize}
\item Problem:
    \begin{itemize}
        \item DataPath: wo liegen die CSV Dateien zum die Daten rausladen
        \item ModelPath: wo soll das Netz hingespeichert werden/Hergeladen - mit Checkpoints usw.
        \item ImagePath: wo sollen Bilder hingespeichert werden, z.B. von Plot
        \item separator: 0 oder 1, jenachdem ob es den nächsten Schritt (0) oder zum Düsenbalken (1) prädizieren soll
    \end{itemize}

    Falls Separator 1:
    \begin{itemize}
        \item separationPosition: Koordinate des Düsenbalken und Ziel der Prädiktion
        \item ThresholdCutoff \todo{verify}
        \item predictionCutOff: Koordinate hinter der keine FeatureTupel mehr genommen werden
    \end{itemize}
\item Train:

\end{itemize}


\subsection{Hyperparameter Tuning}

Als Hyperparameter Optimierung oder auch Hyperparameter Tuning bezeicht man den Vorgang das am besten geeignete Set an 
Hyperparametern für einen Lernalgorithmus zu wählen.


Vorgehen bei dieser Arbeit: Jeweils für NextStep und für Separator getrennte Konfigurationen finden.

Suchen auf Simulierten Daten und dann auf allen Daten verwenden.



Aktueller Stand:
NextStep: kein Overfitting gefunden => L1 und L2 Regularisierung haben keinen positiven Effekt
Adam Optimizer ist am besten
leaky\_relu reigns supreme
Learning Rate Decay ist eine gute Idee. (Höherer Wert => langsamerer Zerfall)
FeatureSize 5 ist gut, weniger wird schlechter und Mehr ist auch schlechter geworden (potenziell weil weniger Beispiele?)

Separator: 

\subsection{Architektur des neuronalen Netzes}

Input layer: \(2 * FeatureSize\) Neuronen

\(N\) hiddenlayer (as determined by Hyperparameter tuning) mit jeweils \(m\) Neuronen.
Fully connected!

Output layer:
Linear activation weil regression.
2 Neuronen, eins für die eine Label Dimension und eins für die andere.

\begin{figure}
	\missingfigure{Grafik architektur}
	% \includegraphics[width=\textwidth]{TrackSortPic}
	\caption{Architektur NN NextStep [TODO Quelle]}
	% \todo{Quelle Bild!}
	\label{fig:netArchitecture}
\end{figure}

    \chapter{Evaluation}

Im vorhergegangenen Kapitel wurde beschrieben, [wie die Netze designed wurde]
Jetzt bewerten wie gut sie das eigentlich mache.

\section{System}

Trainiert wurde auf einem Ubuntu 18.04 Linux System.
Intel i7-7700k CPU @ 4.20 GHz 
NVidia GForce 1080Ti, 11GB GDDR5X
32GB RAM
SSD 

\todo{Stats verifizieren}

\section{Next Step}

Netz Variante 1: den nächsten Schritt vorhersagen
\(\Delta t \) ist immer 1.

definitions vergleichsmodelle:

Notation aus Florian's Diss. Zustandsvector für CV.


\begin{equation*} \label{eq:definitionCV}
    \vx_t = 
    \begin{bmatrix}
        x_t \\
        \dot{x}_t \\
        y_t \\
        \dot{y}_t
       \end{bmatrix} 
\end{equation*}


Zustandsvector für CA:

\begin{equation*} \label{eq:definitionCA}
    \vx_t = 
    \begin{bmatrix}
        x_t \\
        \dot{x}_t \\
        \ddot{x}_t \\
        y_t \\
        \dot{y}_t \\
        \ddot{y}_t
       \end{bmatrix} 
\end{equation*}


\textbf{Constant Velocity Modell}

\begin{equation*} \label{eq:speedCV}
    \dot{\vx}(t) = \mat{A}\vx(t), \quad \mat{A} = 
    \begin{bmatrix}
        0 & 1 & 0 & 0 \\
        0 & 0 & 0 & 0 \\
        0 & 0 & 0 & 1\\
        0 & 0 & 0 & 0
    \end{bmatrix} 
\end{equation*}


\textbf{Contant Acceleration Modell}

\begin{align*} \label{eq:speedCV}
    \dot{\vx}(t) = \mat{A}\vx(t), \quad \mat{A} = 
    \begin{bmatrix}
        \mat{A}_x & \boldsymbol{0} \\
        \boldsymbol{0} & \mat{A}_y
    \end{bmatrix} 
    , \quad
    \mat{A}_x = \mat{A}_y = 
    \begin{bmatrix}
        0 & 1 & 0 \\
        0 & 0 & 1 \\
        0 & 0 & 0
    \end{bmatrix} 
\end{align*}


- CV, CA
- Ergebnis Netz
- Ergebnis Lineare Regression

- Zeit benötigt für evaluation?

\section{Separator}

- CV, CA, (AA?), IA
- Ergebnis Netz
- Ergebnis Lineare Regression
- Ergebnis ausgleichsgerade? Maybe
    \chapter{Fazit und Ausblick}

\todo{einführungstext in Fazit und Ausblick}

\section{Fazit}

Im Rahmen dieser Arbeit wurde gezeigt, dass Neuronale Netze ein Werkzeug sind, mit dem man (was anfangne kann in dem Kontext)

\todo{blick darauf wie es gelaufen ist...}

Mehr daten!

\section{Ausblick}

\todo{was man noch so machen könnte...}

Ende zu Ende lernen: Sollte das Problem mit dem segmentieren lösen, das ich hatte
(Sprengt aber vielleicht den Rahmen einer MA)


    \nocite{*}
	\cleardoublepage
	\phantomsection
	\addcontentsline{toc}{chapter}{Literatur}
    \printbibliography %[heading=bibintoc]


    \appendix
    \chapter{Anhang}

Hier ist der Anhang. Hier kommen Dinge Rein, wie Evaluationsergebnisse, 
die den Hauptteil zu voll machen würden, Tabellen mit daten, die nur begrenzt was mit der Arbeit zu tun haben,


% Please add the following required packages to your document preamble:
% \usepackage{booktabs}
% Please add the following required packages to your document preamble:
% \usepackage{booktabs}



\end{document}

