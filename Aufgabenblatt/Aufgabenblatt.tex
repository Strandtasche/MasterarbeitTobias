\documentclass [a4paper, 10pt]{scrartcl}


% Sprache + Encoding
\usepackage{ucs} % unicode
\usepackage[utf8x]{inputenc}
\usepackage{ngerman} % deutsche Sprache
%\usepackage{textcomp}
\include{AufgabenblattHeader}
	

\Arbeitsart{Masterarbeit}
\Titel{Multi-Target-Tracking in der Schüttgutsortierung}


\Student[B.Sc.]{abc34343}
\Matrikelnummer{abc1234}
\Referent[Prof. Dr.-Ing.]{Uwe D. Hanebeck}
\Mitarbeiter[Dipl.-Inform.]{Florian Pfaff}
\MitarbeiterB[Dr.-Ing.]{Benjamin Noack}
% Beginn Arbeit / Zwischenvortrag / Abgabe
\Termine{15. Dezember 2014}{$\approx$ 15. März 2015}{15. Juni 2015}

% Datum für Unterschrift:
% \renewcommand{\today}{16. Dezember 2013}
\setlength{\parskip}{5pt}

\begin{document}
\Anfang
% \vspace*{}
Bei sogenannten optischen Bandsortieranlagen werden Teilchen auf Basis
visueller Eigenschaften klassifiziert. Bei der Anlage, so wie sie bei
Kunden des Fraunhofer IOSB im Einsatz ist, werden durch richtiges Timing
und gezieltes Aktivieren von Druckluftdüsen eine Klasse Teilchen während
einer, nach der Bandkante beginnender, Flugphase von der anderen separiert.
Aufgrund von Verzögerungen ist es nicht möglich, die Klassifikation und das
Ausblasen gleichzeitig zu vollziehen. Deshalb muss die Position der
Teilchen nach ihrer Klassifikation prädiziert (vorhergesagt) werden. Im
Produktivsystem basiert die Prädiktion auf einer Positionsmessung mittels
einer Zeilenkamera, kombiniert mit der Annahme, dass sich alle Teilchen mit
gleicher Geschwindigkeit in Laufrichtung des Bandes bewegen.

Während diese Annahme bei ausreichender Materialberuhigung bei vielen
Schüttgütern zu guten Ergebnissen führt, gibt es mehrere Schüttgüter, bei
denen die Anforderungen an die Materialberuhigung zu hoch sind. Des
Weiteren wurden Schüttgüter erkannt, bei denen die Klassifikation mittels
der Zeilenkamera nur unzureichende Ergebnisse liefert. 
Zur Lösung dieser beiden Probleme soll eine Flächenkamera genutzt werden, mit Hilfe derer die zu sortierenden Teilchen über
mehrere Zeitschritte hinweg beobachtet werden können. Dies ermöglicht, die
Bewegung der Teilchen besser vorherzusagen und neue Merkmale aus der
Bewegung zu gewinnen.

Im Rahmen dieser Masterarbeit werden Multi-Target-Trackingverfahren untersucht und entwickelt, die zur Verfolgung 
des Schüttgutstroms eingesetzt werden können. Hierzu müssen die durch eine Bandsortieranlage gegebenen Rahmenbedingungen berücksichtigt werden. 
In umgekehrter Richtung werden zudem Kriterien aus den Trackingverfahren abgeleitet, auf deren Basis wichtige Parameter der Sortieranlage, wie z.B. Geometrie und Bandgeschwindigkeit optimiert werden können.  
% Diese Masterarbeit baut auf dem derzeitigen Stand des Experimentalsystems
% auf, erweitert diesen und liefert neue Erkenntnisse, wie gut die
% anvisierten Ziele durch das verbesserte Experimentalsystem erreicht werden
% können.


\textbf{Aufgaben}
\begin{itemize}
% \setlength{\itemsep}{1pt}
  \item Anpassung und Weiterentwicklung von Multi-Target-Trackingverfahren für Schüttgüter in Bandsortieranlagen mit einhergehender Bewertung der Schätzqualität
  \item Bei der Bewertung der Trackingverfahren und der Schätzqualität sollen reale Kameraeffekten, Abtasteffekte, Band- und Objekteigenschaften berücksichtigt werden.
  \item Aus den Trackingverfahren sollen optimale Bedingungen an den Aufbau der Sortieranlage abgeleitet werden. Diese Erkenntnisse sollen in den Aufbau des Experimentalsystems Tablsort 2.0 einfließen.
  \item Die entwickelten Trackingverfahren werden um die Möglichkeit der Parameterschätzung erweitert, so dass objektspezifische Bewegungseigenschaften bestimmt werden können und als Sortierkriterium herangezogen werden können.
\end{itemize}
\Ende
\end{document}

