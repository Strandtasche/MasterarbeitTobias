\documentclass [a4paper, 10pt]{scrartcl}


% Sprache + Encoding
\usepackage{ucs} % unicode
\usepackage[utf8x]{inputenc}
\usepackage{ngerman} % deutsche Sprache
%\usepackage{textcomp}
\include{AufgabenblattHeader}
	

\Arbeitsart{Masterarbeit}
\Titel{Parameterfindung für Kalmanfilter mittels neuronaler Netze}


\Student[B.Sc.]{Tobias Hornberger}
\Matrikelnummer{1697163}
\Referent[Prof. Dr.-Ing.]{Uwe D. Hanebeck}
\Mitarbeiter[Dipl.-Inform.]{Florian Pfaff}
\MitarbeiterB[M.Sc.]{Georg Maier}


% Beginn Arbeit / Zwischenvortrag / Abgabe
\Termine{15. Dezember 2014}{$\approx$ 15. März 2015}{15. Juni 2015}

% Datum für Unterschrift:
% \renewcommand{\today}{16. Dezember 2013}
\setlength{\parskip}{5pt}

\begin{document}
\Anfang
% \vspace*{}
Kalmanfilter sind ein mathematisches Verfahren, dass in vielen Bereichen Anwendung findet. 
Es wird unter anderem in einem optischen Bandsortieranlage des Fraunhofer IOSBs dazu eingesetzt die Bewegung einzelner Schüttgutelemente zu prädizieren.
Um korrekte Vorhersagen zu bestimmen wird ein Bewegungsmodell sowie akkurate Beschreibungen des Mess- sowie des Systemrauschens benötigt.
\textit{Motivation, wie es schwer ist so ein Bewegungsmodell zu finden}.


Maschinelle Lernverfahren haben in der letzter Zeit durch ihre Fähigkeit 
komplexe Muster in großen Datenmengen zu finden ohne diese von Hand modellieren zu müssen an Relevanz gewonnen.
\textit{benötigt große Mengen an Trainings- und Testdaten}

Im Rahmen dieser Masterarbeit soll untersucht werden inwiefern Neuronale Netze das aufwendige Feintuning der  Kalmanfilter-Parameter von Hand ersetzen können.
\textit{Dazu werden ein Haufen Daten gebraucht, die vorhanden nutzen\/neue Suchen}


% Diese Masterarbeit baut auf dem derzeitigen Stand des Experimentalsystems
% auf, erweitert diesen und liefert neue Erkenntnisse, wie gut die
% anvisierten Ziele durch das verbesserte Experimentalsystem erreicht werden
% können.


\textbf{Aufgaben}
\begin{itemize}
% \setlength{\itemsep}{1pt}
  \item Modellieren und trainieren eines neuronalen Netzes, dass die Parameter eines Kalmanfilters ermittelt
  \item \textit{Daten Sammeln und einsetzen}
  \item \textit{Ein Material? Generalisieren?}
  \item 
\end{itemize}
\Ende
\end{document}

