\documentclass [a4paper, 10pt]{scrartcl}


% Sprache + Encoding
\usepackage{ucs} % unicode
\usepackage[utf8x]{inputenc}
\usepackage{ngerman} % deutsche Sprache
%\usepackage{textcomp}
\include{AufgabenblattHeader}
	

\Arbeitsart{Masterarbeit}
\Titel{Ableitung von Bewegungsmodellen für Anwendungen in der Schüttgutsortierung mittels Machine Learning} 
%Parameterfindung für Kalmanfilter mittels Neuronaler Netze


\Student[B.Sc.]{Tobias Hornberger}
\Matrikelnummer{1697163}
\Referent[Prof. Dr.-Ing.]{Uwe D. Hanebeck}
\Mitarbeiter[Dipl.-Inform.]{Florian Pfaff}
\MitarbeiterB[M.Sc.]{Georg Maier}
\MitarbeiterC[Dr.-Ing.]{Benjamin Noack}
\MitarbeiterD[Dr.-Ing.]{Robin Gruna}

% Beginn Arbeit / Zwischenvortrag / Abgabe
\Termine{22. Juni 2018}{$\approx$ 1. September 2018}{21. Dezember 2018}

% Datum für Unterschrift:
% \renewcommand{\today}{16. Dezember 2013}
\setlength{\parskip}{5pt}

\begin{document}
\Anfang
% \vspace*{}
Kalman-Filter sind ein mathematisches Verfahren, das in vielen Bereichen Anwendung findet. 
Es wird unter anderem in optischen Bandsortieranlagen des Fraunhofer IOSBs dazu eingesetzt, Bewegungen einzelner Schüttgutelemente zu prädizieren.
Um korrekte Vorhersagen zu erhalten, wird ein Bewegungsmodell sowie akkurate Beschreibungen des Mess- sowie des Systemrauschens benötigt.
Ein gutes Bewegungsmodell zu bestimmen ist aufwendig und verschiedene Bewegungsmodelle erreichen bei unterschiedlichen Schüttgütern Ergebnisse mit unterschiedlicher Qualität.
%\textit{[Motivation, wie es schwer ist so ein Bewegungsmodell zu finden]}.


Maschinelle Lernverfahren haben in letzter Zeit durch ihre Fähigkeit, 
komplexe Muster in Datensätzen zu finden, ohne weitreichende händische Vorgaben machen zu müssen, an Relevanz gewonnen.
Speziell neuronale Netze erleben eine Renaissance und werden für eine große Menge an unterschiedlichsten Problemen eingesetzt.
Hierfür werden große Mengen an Trainings- und Testdaten benötigt. 
%\textit{[benötigt große Mengen an Trainings- und Testdaten]}

%Im Rahmen dieser Masterarbeit soll untersucht werden, inwiefern Neuronale Netze das aufwendige Feintuning der Kalmanfilter-Parameter von Hand ersetzen können.
Im Rahmen dieser Masterarbeit sollen verschiedene Ansätze untersucht werden, wie neuronale Netze eingesetzt werden können, um die Bewegung von Schüttgutpartikeln zu prädizieren.
Dazu müssen vorhandene Daten von der Schüttgutsortierung aufbereitet werden und neue Daten gesammelt werden.
Nachdem ein Netzwerk mit Hilfe der Tensorflow-Software-Library modelliert wurde, kann dieses dann mit diesen Daten trainiert werden.

%\textit{[Dazu werden ein Haufen Daten gebraucht, die vorhanden nutzen\/neue Suchen]}


% Diese Masterarbeit baut auf dem derzeitigen Stand des Experimentalsystems
% auf, erweitert diesen und liefert neue Erkenntnisse, wie gut die
% anvisierten Ziele durch das verbesserte Experimentalsystem erreicht werden
% können.


\textbf{Aufgaben}
\begin{itemize}
% \setlength{\itemsep}{1pt}
  \item Datensammlung mittels Schüttgutsortierer des Fraunhofer IOSBs \\sowie Datenvorverarbeitung 
  \item Data-Augmentation der Schüttgutdaten
  %\item Modellieren und trainieren eines neuronalen Netzes, dass die Parameter eines Kalman-Filters ermittelt
  \item Erproben von verschiedenen Ansätzen für die Bewegungsprädiktion von Schüttgutelementen mittels neuronalen Netzen
  	\begin{itemize}
  		\item Modellieren und Trainieren verschiedener Netze
  		\item Vergleich der verschiedenen Ansätze mit dem State of the Art und gegebenenfalls untereinander
  		%\item [Vergleich der verschiedenen Ansätze]
  	\end{itemize}
\end{itemize}
\Ende
\end{document}

